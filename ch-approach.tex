% vim:spell spelllang=fr
\chapter{Transformations de modèles par réécriture}
\label{ch:approach}
%15p

%\todo{
%\begin{itemize}
%  \item Représentation de modèles dans le monde des grammaires. Il faut faire
%  une petite partie de préambule pour expliquer que l'on amène les modèles dans
%  le monde des grammaires pour utiliser des outils/techniques existant.
%  Certains transposent les techniques dans le monde des modèles.  nous :
%  \begin{itemize}
%    \item métamodèle $\leftrightarrow$ signature ; 
%    \item modèle $\leftrightarrow$ terme ;
%    \item génération d'une signature à partir d'un MM
%    \item création de termes représentant un modèle / ou chargement d'un modèle
%    directement mappé comme un terme $\rightarrow technique, non ? dans l'autre
%    partie ?$
%    \item transformation d'un terme (représentant donc une transformation de
%    modèle) $\rightarrow$ oui, c'est l'approche, trivial
%    \item le terme résultant peut ensuite être utilisé comme tout modèle
%    (mapping cible) $\rightarrow$ oui, trivial quand on a expliqué l'approche
%  \end{itemize}
%  \item Décomposition = approche compositionnelle
%  \item Résolution - réconciliation
%  \item Outil pour exprimer une transformation de modèle en Tom
%  \begin{itemize}
%    \item Extension du langage Tom
%      \begin{itemize}
%        \item syntaxe : \lex{\%transformation}, \lex{\%resolve}, \lex{\%tracelink}
%        \item illustration :??? ou alors on met le cas d'utilisation ici pour
%        expliquer ?
%      \end{itemize}
%    \item{Évolutions}
%    \begin{itemize}
%      \item[Modularité : ] cf. resolve modulaire, j'ai un proto qui est fait,
%      mais pas tant de réflexion que ça
%      \item[Traçabilité : ] ici je ne parle pas trop de la traçabilité liée aux
%      spéc mais plus la traçabilité technique/interne. Le "\%resolvelink" de
%      \%tracelink, + éventuellement une traçabilité type "log complet"
%      \item[Expressivité : ] simplification de syntaxe (traversal, arguments,
%      patterns, multi sources et cibles), lien avec le travail de Christophe ?
%    \end{itemize}
%  \end{itemize}
%  \item Travaux connexes : approche classique en MDE, outils, différences
%\end{itemize}}

%\todo{approche classique, à la QVT (resolve) ; à mettre ; Mais ce qui est moins classique : côté hybride 1. entre DSL et GPL ; 2. usages
%d'outils d'un espace technologique pour appliquer des transformations de
%modèles d'un autre espace technologique.}

Dans ce chapitre, nous expliquons l'approche que nous avons adoptée pour mettre
en œuvre les transformations de modèles. Dans la
section~\ref{approach:sec:hybride}, nous présentons l'aspect hybride de notre
approche, les choix liés ainsi que son intérêt. Nous abordons ensuite la
problématique de la représentation des modèles dans notre approche dans la
section~\ref{approach:sec:representation}. Nous expliquons dans la
section~\ref{approach:sec:approche} les mécanismes en jeu pour notre approche
de transformation de modèles par réécriture.

%\ttodo{
%  \begin{itemize}
%    \item approche hybride : avantages + transfo par réécriture changement
%      d'espace technique ; hybride car GPL+DSL, changement d'espace technique,
%      non lié à une techno même si la première implém' l'est (extensible)
%    \item représentation de modèles
%    \item explication de l'approche classique décomposition+réconciliation avec
%      les \emph{placeholders}
%    \item avantages : modularité, traçabilité, expressivité
%  \end{itemize}
%}

\section{Choix et intérêt d'une approche hybride}
\label{approach:sec:hybride}

Si le principe de la transformation elle-même est un principe qui peut sembler
classique, car adopté par d'autres outils tels que {\qvt} ou {\atl}, notre
environnement de travail est très différent de celui des autres outils, ce qui
fait l'originalité de notre approche. %En effet, nous nous plaçons dans un
%environnement technique au croisement de plusieurs langages et technologies.

Comme nous l'avons vu dans le chapitre~\ref{ch:notions}, il existe de multiples
approches pour implémenter des transformations de modèles. L'un des choix
possibles est d'opter pour une approche par programmation en utilisant un
langage généraliste tel que {\java}. On lui adjoint un \emph{framework} et un
ensemble de bibliothèques pour faciliter la représentation et la manipulation
de modèles, notamment {\emf}. Ensuite, la transformation revient à l'écriture
d'un programme qui charge un modèle en entrée et produit un modèle en sortie.
Cette approche présente certains avantages. Les langages généralistes sont
habituellement plus accessibles à la plupart des développeurs tant par le fait
qu'une formation d'informaticien comprend généralement l'apprentissage de
langages généralistes que par le fait que ces langages suivent des paradigmes
de programmation bien identifiés. En outre, cette plus grande accessibilité a
aussi pour effet de faciliter le développement de communautés, ainsi que
d'outils d'aide au développement (IDE, bibliothèques, etc.) et de
documentation. Ces effets entretiennent alors le cercle vertueux de la facilité
d'accès. À l'opposé, il peut être difficile ou long d'implémenter une
transformation complexe alors qu'un langage dédié proposerait les constructions
adéquates pour ce type de tâche.

Une deuxième possibilité est de suivre les recommandations {\mda} qui
encouragent l'utilisation de {\dsl}. Cela a l'avantage d'être exactement adapté
à la tâche visée. L'inconvénient est que les {\dsl} ont généralement une
visibilité plus limitée, avec des communautés plus petites et un outillage
moins étoffé. De plus, la compétence est plus difficile à trouver ou induit des
coûts de formation plus élevés.

Notre approche est une approche hybride, pour combler le fossé entre les
langages généralistes et dédiés, afin de bénéficier du meilleur des deux
mondes. Nous proposons d'utiliser le langage {\tom} pour exprimer des
transformations de modèles, ce qui nous permet de nous intégrer dans des
langages généralistes tout en apportant des fonctionnalités supplémentaires par
ses constructions. Comme présenté dans le chapitre~\ref{ch:tom}, {\tom} repose
sur le calcul de réécriture et implémente la fonctionnalité de \emph{stratégie
de réécriture}, ce qui permet de découpler les règles métier du parcours des
structures de données. Nous proposons donc de transformer des modèles par
réécriture, en nous appuyant sur les stratégies. Manipulant des termes, il nous
faut représenter les modèles sous cette forme pour pouvoir les transformer.
Nous avons aussi fait le choix d'une intégration dans un environnement de
production pouvant être considéré comme typique, à savoir l'écosystème {\java}.
De plus, nous avons choisi de travailler dans un premier temps avec une
technologie courante et très utilisée ---~{\emf}~--- qui devient un standard
\emph{de fait}.

Partant de ces choix et de nos contraintes, notre approche consiste à exprimer
un modèle {\emf} sous la forme d'un terme puis à le transformer par réécriture.
Nous opérons donc des transformations de modèles en changeant d'espace
technologique pour pouvoir nous appuyer sur nos méthodes et outils déjà
éprouvés. Nous décrivons le premier aspect de notre approche ---~la
représentation des modèles~--- dans la
section~\ref{approach:sec:representation}, puis nous détaillons les principes
mis en œuvre dans une transformation dans la
section~\ref{approach:sec:approche}.

%Notre approche consi
%
%En effet, dans un contexte industriel
%contexte économique, réduction des coûts
%et la représentation choisie pour les modèles porte à conséquences sur les outils.

\section{Représentation de modèles par une signature algébrique}
\label{approach:sec:representation}

%\ttodo{ici on explique l'on représente les modèles sous forme de termes. On a
%donc aussi des MM, mais cela se présente sous la forme d'une signature (ex :
%gom vs .ecore}

Réécrivant des termes, la première étape de notre approche est d'élaborer une
représentation adéquate des modèles. Compte tenu de ce besoin et de notre
approche, nous procédons en premier lieu par à un changement d'espace
technologique. Un métamodèle dans le monde des modèles correspond à une
signature dans celui des termes, tandis qu'un modèle ---~conforme à son
métamodèle~--- correspond à un terme ---~conforme à sa signature
algébrique~---. Lors de ce changement, un aspect \emph{a priori} problématique
de notre représentation sous forme de terme d'un modèle vient du fait qu'un
terme est une structure arborescente tandis qu'un modèle est généralement vu
comme un graphe (un ensemble de nœuds et d'arcs). Une transformation de modèle
est donc une fonction de transformation de graphe qu'il faut encoder dans notre
contexte. L'un des aspects de notre approche est d'opérer une transformation
préalable permettant de considérer une transformation de modèles comme une
transformation de termes. Cela est possible étant donné que l'on obtient un
arbre de recouvrement par la relation de composition, ce qui nous permet
d'établir une vue adéquate des modèles sous la forme d'un terme.

Nous avons donc développé un outil ---~appelé {\tomemf}~--- que nous décrivons
techniquement dans le chapitre~\ref{ch:outils}, et qui, pour un métamodèle
{\emf} {\ecore} donné, produit sa signature algébrique {\tom} correspondante,
utilisable dans un programme {\tomjava}. Bien que cet outil repose pour le
moment sur une technologie particulière, il est tout à fait possible de
l'étendre ou d'écrire un outil similaire adapté à d'autres technologies.

Dans notre représentation, pour chaque métaclasse, une sorte lui correspond,
ainsi qu'un opérateur. Une métaclasse pouvant avoir des attributs et des
opérations, l'opérateur a les champs correspondants. Les relations sont
représentées par des attributs simples dans le cas d'associations. Dans le cas
de relations de composition, des sortes et opérateurs variadiques additionnels
sont créés, et un attribut de ce type est ajouté à l'opérateur ayant la
relation de composition.

Pour illustrer notre propos, considérons un métamodèle simple permettant de
décrire des graphes (figure~\ref{fig:graphmm}). Un graphe (\emph{Graph}) est
composé de \emph{Nodes} et d'\emph{Arcs}. Chaque arc a un poids, une source et
une cible. Ces extrémités sont de type \emph{Node}, un \emph{Node} pouvant
avoir plusieurs arcs entrants et sortants.
\begin{figure}[h]%[fig:graphmm]{Métamodèle de graphe.}%[H] %[!h]
  \begin{center}
    %\begin{tikzpicture}[scale=0.7,transform shape]
\begin{tikzpicture}[scale=1,transform shape]

  \begin{class}{Graph}{0,0}
  \end{class}
  
  \begin{class}{Arc}{3,-3}
    \attribute{weight : Int}
  \end{class}
  
  \begin{class}{Node}{-3,-3}
    \attribute{name : String}
  \end{class}
  
  \composition{Graph}{nodes}{*}{Node}
  \unidirectionalAssociation{Node}{graph}{1}{Graph}
  
  \composition{Graph}{arcs}{*}{Arc}
  \unidirectionalAssociation{Arc}{graph}{1}{Graph}

  %\association{Arc}{source}{1}{Node}{0..*}{outgoings}
  \myassociation{Node}{source}{1}{Arc}{*}{outgoings}{0,-3.0}{0}

  %\association{Arc}{target}{1}{Node}{0..*}{incomings}
  \myassociation{Node}{target}{1}{Arc}{*}{incomings}{0,-4.5}{1}

\end{tikzpicture}

    \caption{Métamodèle des graphes.}
    \label{fig:graphmm}
  \end{center}
\end{figure}
La représentation sous forme de signature algébrique de ce métamodèle est alors
donnée par le listing~\ref{code:signalgGraph}. Les métaclasses \emph{Graph},
\emph{Node} et \emph{Arc} ont chacune une sorte qui lui correspond (nous avons
conservé les même noms). Un opérateur ---~préfixé par \texttt{op} dans
l'exemple~--- est associé à chacune d'entre elles. Les attributs présents dans
les métaclasses sont bien reportés au niveau des opérateurs (\texttt{name} et
\texttt{weight}). Les relations d'association sont traduites par des paramètres
additionnels : l'opérateur \texttt{opArc} possède ainsi deux paramètres
supplémentaires \texttt{source} et \texttt{target}. Le cas des relations de
composition (relations \texttt{nodes} et \texttt{arcs}) est traité par la
création d'opérateurs variadiques (\texttt{opNodeList} et \texttt{opArcList}
dans notre exemple) ainsi que de leurs sortes associées (\texttt{NodeList} et
\texttt{ArcList}).

\begin{figure}[H]
  \centering
  \begin{gomcode}[caption=Signature algébrique correspondant au métamodèle de la
  figure~\ref{fig:graphmm}.,label=code:signalgGraph]
Graph = opGraph(nodes:NodeList, arcs:ArcList)

Node  = opNode(name:String, graph:Graph, incomings:ArcList, outgoings:ArcList)

Arc   = opArc(weight:Int, graph:Graph, source:Node, target:Node)

NodeList = opNodeList(Node*)

ArcList   = opArcList(Arc*)
\end{gomcode}


%  \caption{}
%  \label{code:signalgGraph}
\end{figure}


Grâce à ce changement d'espace technologique, nous encodons la fonction qui
transforme un modèle non plus comme une fonction de réécriture de graphe, mais
comme une fonction de réécriture de termes.


%Dans notre approche de transformation par réécriture, nous représentons les
%métamodèles sous la forme de signature algébrique. Plutôt que de parler de
%métaclasses et de métarelations, nous parlons 
%
%\ttodo{signature $\leftrightarrow$ métamodèle ; terme $\leftrightarrow$ modèle
%; métaclasses contiennent des operations et des attributs ; peuvent être liées
%par des métarelations (association ou composition)}

\FloatBarrier

\section{Transformation de modèles par réécriture}
\label{approach:sec:approche}

%\todo{à bouger}

%Étant en mesure de représenter des modèles sous forme de termes ---~et donc de
%voir une transformation de modèles non plus comme de la réécriture de graphe
%mais comme de la réécriture de terme, nous pouvons
Étant en mesure de représenter des modèles sous forme de termes, nous pouvons
décrire notre approche pour les transformer. Son principe est similaire à celui
de l'approche de {\qvt} et {\atl} et peut sembler classique dans le domaine des
modèles. Toutefois, ce n'est pas le cas dans l'environnement dans lequel nous
évoluons, et plus généralement dans le domaine des outils de transformation
généralistes dédiés aux arbres. Habituellement, à l'instar d'un compilateur, les
outils de transformation d'arbres procèdent à des parcours et à des
modifications successives sur un arbre qui est passé de phase en phase de
l'outil.

Dans notre approche, nous décomposons les transformations en deux phases
distinctes. La première est une transformation par parties qui consiste à créer
les éléments cibles du modèle résultant ainsi que des éléments additionnels que
nous appelons \emph{éléments resolve}, en référence au \emph{resolve} de {\qvt}
et au \emph{resolveTemp} d'{\atl}. Ces éléments permettent de faire référence à
des éléments qui n'ont pas encore été créés par la transformation. La seconde
phase, quant à elle, a pour objectif de rendre le modèle cible résultat
cohérent, c'est-à-dire conforme au métamodèle cible en éliminant les éléments
\emph{resolve} et en les remplaçant par des références vers les éléments
effectivement créés par la transformation. Cette seconde phase n'ajoute aucun
nouvel élément cible au résultat.

Pour illustrer notre propos dans ce chapitre, nous nous appuierons sur un
exemple visuel permettant de bien comprendre le mécanisme de
\emph{décomposition-résolution}. Supposons que nous souhaitons transformer une
séquence \texttt{A;B} textuelle en sa forme graphique correspondante comme
décrit par la Figure~\ref{fig:simpleApproachExample}. Dans cet exemple, le
choix des couleurs des connecteurs est arbitraire : nous aurions très bien pu
choisir de colorer le cercle en vert et le carré en bleu. Supposons donc que ce
découpage est spécifié et imposé.

\begin{figure}[h]
  \begin{center}
   %\begin{figure}[h!]

  \definecolor{myred}{HTML}{d01e1e}
  \definecolor{mygreen}{HTML}{129d1c}
  \definecolor{myblue}{HTML}{0000FF}
  \makeatletter

%\begin{center}
      %\resizebox{10cm}{!}{
  \begin{tikzpicture}[>=latex, node distance=1cm, on grid, auto]

    \node (S1) [draw, regular polygon, regular polygon sides=3, minimum
    size=1cm, shape border rotate=180, color=mygreen] {}; %{;};

    \path (S1.west)+(-0.5,-1) node (A2) [draw, circle, minimum size=0.5mm, color=myred] {};
    \path (A2.west)+(-0.25,-1) node (A1) [draw, regular polygon, regular
    polygon sides=6, minimum size=1cm, color=myred] {}; %{A};

  \path (S1.east)+(0.5,-1) node (S2) [draw, rectangle, minimum size=0.5mm, color=mygreen] {};
    \path (S2.east)+(0.25,-1) node (B1) [draw, regular polygon, regular polygon
    sides=5, minimum size=1cm, color=myblue] {}; %{B};

       %\node (A1) [draw, regular polygon, regular polygon sides=6, minimum size=1cm] {A};
       %\path (A1.east)+(0.25,1) node (A2) [draw, circle, minimum size=0.5mm] {};
       %\path (A2.east)+(0.5,1) node (S1) [draw, regular polygon, regular polygon sides=3, minimum size=0.5cm, shape border rotate=180] {;};
       %\path (S1.east)+(0.5,-1) node (S2) [draw, rectangle, minimum size=0.5mm] {};
       %\path (S2.east)+(0.25,-1) node (B1) [draw, regular polygon, regular polygon sides=5, minimum size=1cm] {B};

    \path[-,color=myred] (A1) edge (A2);
    \path[-,color=mygreen] (A2) edge (S1);
    \path[-,color=mygreen] (S1) edge (S2);
    \path[-,color=myblue] (S2) edge (B1);

    \path (A2.west)+(-1.5,0) node (arrow) [minimum size=0.5cm] {$\longrightarrow$};
    \path (arrow.west)+(-1,0) node (AB) [draw, rectangle, inner sep=0.2cm] {A;B};

  \end{tikzpicture}
%}
  %\caption{}
%\label{fig:simpleApproachExample}
%\end{center}
%\end{figure}

   \caption{Transformation du modèle source \texttt{A;B} en un modèle cible graphique.}
    \label{fig:simpleApproachExample}
  \end{center}
\end{figure}

%Chacune de ces phases peut être vue comme une fonction que nous composons par
%la suite pour former une transformation. 

%\noindent En instanciant notre approche avec le cas d'étude SimplePDLToPetriNet, nous
%obtenons : %$SimplePDLToPetriNet$ comme la composée de $Transformer$ et de
%%$Resolve$, avec $MM_{SimplePDL}$, $MM_{PetriNet_{resolve}}$ et $MM_{PetriNet}$
%%respectivement les métamodèles source, étendu et cible :
%\begin{tabbing}
%  $SimplePDLToPetriNet$ \= $ : MM_{SimplePDL} \rightarrow MM_{PetriNet}$\\
%  $Transformer$ \> $ : MM_{SimplePDL} \rightarrow MM_{PetriNet_{resolve}}$\\
%  $Resolve$ \> $ : MM_{PetriNet_{resolve}} \rightarrow MM_{PetriNet}$\\
%  $SimplePDLToPetriNet $ \> $ = Resolve \circ Transformer$\\
%\end{tabbing}
%Appliquée au processus $p_{root}$ (illustré Figure~\ref{fig:simplepdlusecase})
%conforme à $MM_{SimplePDL}$ (Figure~\ref{fig:simplepdlmmodel}), cette
%transformation produira un réseau de Petri $pn$ (illustré
%Figure~\ref{fig:petrinetusecase}) conforme à $MM_{PetriNet}$
%(Figure~\ref{fig:petrinetmmodel}), en passant par le résultat intermédiaire
%$pn_{resolve}$ conforme au métamodèle $MM_{PetriNet_{resolve}}$. On obtient
%donc :
%\begin{flushleft}
%  $SimplePDLToPetriNet(p_{root}) = Resolve(Transformer(p_{root}))$, avec\\
%  $Transformer(p_{root}) = pn_{resolve}$ et\\ 
%  $Resolve(p_{resolve}) = pn$
%\end{flushleft}
%La Figure~\ref{fig:mmresolveinst} instancie le schéma d'extension du
%métamodèle cible au cas d'étude : le métamodèle cible est enrichi
%d'une métaclasse \emph{ResolvePT} pour pouvoir créer des éléments
%intermédiaires \emph{resolve} qui jouent temporairement le role d'une
%\emph{Transition} obtenue à partir d'une source \emph{Process}.
%\begin{figure}[h]
%  \begin{center}
%    \begin{tikzpicture}[node distance=1.1cm,>=stealth',bend angle=45,auto,scale=1,transform shape]
  \tikzstyle{every label}=[black]
  \begin{scope}

    \node (MMS) {$MM_{Text}$};

    \node (MMSA) [right of=MMS,xshift=0.5cm] {} ;
    \node (MMSC) [right of=MMS,xshift=-0.2cm] {} ;

    %\node (MMRes) [right of=MMS,xshift=1cm] {$MM_{t_{resolve}}$};
    \node (MMRes) [right of=MMSA,xshift=0.5cm] {$MM_{Picture_{resolve}}$};
    
    \node (MMTA) [right of=MMRes,xshift=0.5cm] {} ;
    \node (MMTC) [right of=MMRes,xshift=0.2cm] {} ;
    
    %\node (MMT) [right of=MMRes,xshift=1cm] {$MM_t$};
    \node (MMT) [right of=MMTA,xshift=0.5cm] {$MM_{Picture}$};
    
    \node[draw,rectangle,inner sep=0.1cm] (S) [below of=MMS] {$A$};
    \node[draw,rectangle,inner sep=0.1cm] (TRes) [below of=MMRes] {$Circle$};
    %\node[draw,rectangle,inner sep=0.1cm] (TResRes) [below of=TRes] {Resolve$E_s^jE_t^i$};
    \node[draw,rectangle,inner sep=0.1cm] (TResRes) [below of=TRes] {$ResolveACircle$};
    \path (TResRes) edge [post] (TRes);
    \node[draw,rectangle,inner sep=0.1cm] (T) [below of=MMT] {$Circle$};

    \node (MMSB) [left of=TResRes,xshift=-0.5cm] {} ;
    \node (MMSD) [left of=TResRes,xshift=-0.2cm] {} ;
    \node (MMTB) [right of=TResRes,xshift=0.5cm] {} ;
    \node (MMTD) [right of=TResRes,xshift=0.2cm] {} ;

    \path (MMSA) edge [dash pattern=on 2pt off 2pt] (MMSB);
    \path (MMTA) edge [dash pattern=on 2pt off 2pt] (MMTB);
    %\path (MMSC) edge [dash pattern=on 2pt off 2pt] (MMSD);
    %\path (MMTC) edge [dash pattern=on 2pt off 2pt] (MMTD);

  \end{scope}
\end{tikzpicture}

%    \caption{Instanciation du schéma d'extension du métamodèle cible pour le cas d'étude SimplePDLToPetriNetPetriNet.}
%    \label{fig:mmresolveinst}
%  \end{center}
%\end{figure}




\subsection{Approche compositionnelle}%\todo{Transformations élémentaires} vs \todo{(OLD) Approche compositionnelle}}
\label{approach:subsec:composition}

%\todo{Note : Décomposition en transformations élémentaires == définitions,
%ajouter la description de  l'exemple "A ; B" ?}

Une fois le problème de la représentation des modèles résolu, il est possible
d'instancier un modèle (création d'un terme conforme à la signature algébrique)
et d'opérer une transformation sur le terme le représentant.

L'écriture d'une transformation de modèles peut se faire par une approche
procédurale \emph{monolithique}. L'utilisateur construit la transformation par
étapes (\emph{transformation steps}), dont l'ordre s'impose naturellement
%\ttodo{non, détailler, exemple à illustrer visuellement}
en fonction des besoins des différents éléments : par exemple, la
transformation décrite dans la Figure~\ref{fig:simpleApproachExample} peut se
décomposer en trois étapes que nous illustrons dans la
Figure~\ref{fig:approachSimpleRules}.

\begin{figure}[h]
  \centering
\begin{tabular}{c|c|c}
  %\begin{subfigure}[A]{0.30\textwidth}
  \begin{subfigure}{0.30\textwidth}
    \centering
    \definecolor{myred}{HTML}{d01e1e}
\begin{tikzpicture}[>=latex, node distance=1cm, on grid, auto]%, scale=0.6, transform shape]

\node (A2) [draw, circle, minimum size=0.5mm, color=myred] {};
\path (A2.west)+(-0.25,-1) node (A1) [draw, regular polygon, regular
polygon sides=6, minimum size=1cm, color=myred] {}; %{A};
\path[-,color=myred] (A1) edge (A2);

\path (A2.west)+(-1.5,-0.5) node (arrow) [minimum size=0.5cm] {$\longrightarrow$};
\path (arrow.west)+(-0.5,0) node (A) {A};

\end{tikzpicture}

    \subcaption{}
  \end{subfigure}
  &
  %\begin{subfigure}[seq]{0.30\textwidth}
  \begin{subfigure}{0.30\textwidth}
    \centering
    \definecolor{mygreen}{HTML}{129d1c}
\begin{tikzpicture}[>=latex, node distance=1cm, on grid, auto]

\node (S1) [draw, regular polygon, regular polygon sides=3, minimum
size=1cm, shape border rotate=180, color=mygreen] {}; %{;};
\path (S1.west)+(-0.5,-1) node (A2) [circle, minimum size=0.5mm] {};
\path (S1.east)+(0.5,-1) node (S2) [draw, rectangle, minimum size=0.5mm, color=mygreen] {};

\path[-,color=mygreen] (A2) edge (S1);
\path[-,color=mygreen] (S1) edge (S2);

\path (S1.west)+(-1,-0.25) node (arrow) [minimum size=0.5cm] {$\longrightarrow$};
\path (arrow.west)+(-0.5,0) node (seq) {;};

\end{tikzpicture}

    \subcaption{}
  \end{subfigure}
  &
  %\begin{subfigure}[B]{0.30\textwidth}
  \begin{subfigure}{0.30\textwidth}
    \centering
    \definecolor{myblue}{HTML}{0000FF}
\begin{tikzpicture}[>=latex, node distance=1cm, on grid, auto]%, scale=0.6, transform shape]

\node (S2) [rectangle, minimum size=0.5mm] {};
\path (S2.east)+(0.25,-1) node (B1) [draw, regular polygon, regular polygon
sides=5, minimum size=1cm, color=myblue] {}; %{B};

\path[-,color=myblue] (S2) edge (B1);

\path (S2.west)+(-1,-0.5) node (arrow) [minimum size=0.5cm] {$\longrightarrow$};
\path (arrow.west)+(-0.5,0) node (B) {B};

\end{tikzpicture}

    \subcaption{}
  \end{subfigure}
\end{tabular}
  \caption{Règles de transformation de \texttt{A}, \texttt{;} et \texttt{B}.}
  \label{fig:approachSimpleRules}
\end{figure}

La transformation de \texttt{A} (étape (a)) donne un hexagone et un cercle
rouges connectés, celle de \texttt{B} (étape (c)) un pentagone et un
connecteur bleus, celle de \texttt{;} (étape (b)) produit un triangle
et un carré connectés verts, ainsi qu'un connecteur supplémentaire. Pour qu'un
connecteur ou arc puisse être créé, il est nécessaire de connaître chaque
extrémité. Ainsi, la transformation de \texttt{A} ne pose aucun souci
particulier : un seul arc est créé entre deux éléments créés (hexagone et
cercle) dans cette même étape de transformation. En revanche, la
transformation de \texttt{B} en un pentagone est censée aussi construire un arc
dont l'une des extrémités (petit carré) n'est pas créée dans cette étape de
transformation. Il est donc nécessaire que l'étape de transformation
construisant cette deuxième extrémité se déroule avant celle produisant l'arc
(c). Le carré servant de seconde extrémité à l'arc est construit dans l'étape
de transformation de \texttt{;} qui devra donc être exécutée avant (c). Nous
notons que cette étape génère un autre arc dont l'une des extrémités n'est pas
connue dans (b). L'étape de transformation produisant cette extrémité d'arc
qui nous intéresse (petit cercle) est (a). Il est donc naturel d'exécuter (a)
avant (b). Finalement, pour que cette transformation puisse être exécutée sans
qu'il n'y ait de problème d'éléments manquant, l'utilisateur doit adopter les
étapes de transformation dans l'ordre suivant : (a), puis (b), puis (c). S'il
ne respecte pas cet ordre, il sera confronté au problème de création d'éléments
avec des informations manquantes.

Cependant, cette approche n'est pas toujours possible, notamment lorsque l'on
manipule des structures cycliques. Lorsqu'elle est possible, elle nécessite une
parfaite expertise ainsi qu'une connaissance globale de la transformation pour
être capable d'organiser les différentes étapes. De plus, avec une telle
méthode une transformation sera généralement monolithique. Elle sera donc peu
générique et le code peu réutilisable, l'encodage du parcours du modèle ainsi
que les transformations étant {\adhoc}. Généralement, le parcours du modèle
sera encodé par des boucles et de la récursivité, et un traitement particulier
sera déclenché lorsqu'un élément donné sera détecté. Parcours et traitement
seront donc étroitement liés. La moindre modification du métamodèle source ou
cible implique donc de repenser la transformation.

Nous souhaitons au contraire faciliter le développement et la maintenance du
code que l'utilisateur écrit pour une transformation, tout en le rendant
réutilisable pour une autre transformation. Il est donc important d'adopter
une méthode permettant une grande modularité du code.

Notre approche est d'opérer une transformation par parties : il faut d'abord
décomposer une transformation complexe en transformations les plus simples (ce
que nous nommons \emph{transformations élémentaires} ou \emph{définitions}).
Chacune d'entre elles est décrite par une règle ou un ensemble de règles. La
transformation globale est ensuite construite en utilisant ces transformations
élémentaires. Mais dans ce cas se pose le problème de la dépendance des
définitions entre elles, ainsi que de l'utilisation d'éléments issus d'une
transformation élémentaire dans une autre transformation élémentaire. Ce qui a
des conséquences sur l'ordre d'application des définitions : il peut être
absolument nécessaire qu'une partie de la transformation soit effectuée pour
que les autres étapes puissent être appliquées. De plus, par souci
d'utilisabilité, nous ne souhaitons pas que l'utilisateur ait besoin de se
soucier de l'ordre d'application des transformations élémentaires. Nous
souhaitons qu'il se concentre uniquement sur la partie métier de la
transformation. Il faut donc mettre en œuvre un mécanisme permettant de
résoudre les dépendances.

Dans notre contexte, nous effectuons des transformations dites
\emph{out-place}, ce qui signifie que le modèle source n'est pas modifié. Le
modèle cible résultant est construit au fur et à mesure de la transformation,
et n'est pas obtenu par modifications successives du modèle source
---~transformation \emph{in-place}, comme le font les outils
VIATRA~\cite{Varro2002} / VIATRA2~\cite{Varro2007} et
GrGen.NET~\cite{Jakumeit2010} par exemple. Partant de ce constat, les
transformations élémentaires composant notre transformation n'entretiennent
aucune dépendance dans le sens où la sortie d'une transformation élémentaire
n'est pas l'entrée d'une autre transformation élémentaire. Dans notre approche
compositionnelle, chaque sortie d'une transformation élémentaire est une partie
du résultat final.

%\todo{[maintenant, faut parler des \emph{éléments resolve}}

Dans l'exemple, nous conservons la décomposition proposée dans la
Figure~\ref{fig:approachSimpleRules} en trois règles simples. Nous avons
décomposé une transformation complexe en \emph{définitions} indépendantes et
nous pouvons les appliquer. Subsiste cependant le problème de l'usage dans une
\emph{définition} d'éléments créés dans une autre \emph{définition}. Comme
l'ordre d'écriture et d'application des transformations élémentaires ne doit
pas être une contrainte pour l'utilisateur, nous avons choisi de résoudre ce
problème par l'introduction d'éléments temporaires ---~éléments dits
\emph{resolve}~--- qui font office d'éléments cibles durant la transformation,
et qui sont substitués en fin de transformation lors d'une seconde phase. Cette
dénomination fait évidemment référence au \emph{resolve} de {\qvt} et au
\emph{resolveTemp} de {\atl}.

Partant du principe que toutes les transformations élémentaires peuvent être
déclenchées indépendamment dans n'importe quel ordre (voire en parallèle), il
faut être en mesure de fournir un élément cible lorsque le traitement d'une
\emph{définition} le nécessite. Nous proposons donc de construire un terme
temporaire représentant l'élément final qui a été ou qui sera construit lors de
l'application d'une autre \emph{définition}. Ce terme doit pouvoir être intégré
dans le modèle cible temporaire pour être manipulé en lieu et place du terme
ciblé censé être construit dans une autre \emph{définition}. Il doit donc
respecter les contraintes de types du métamodèle cible tout en portant des
informations supplémentaires telles qu'un identifiant, une référence à
l'élément source d'origine et une référence à l'élément cible. Nous étendons
donc le métamodèle cible $MM_t$ afin que ces contraintes soient respectées et
que le modèle intermédiaire résultant soit conforme à un métamodèle cible
étendu, noté $MM_{t_{resolve}}$. Ainsi, tout élément \emph{resolve}
$e_{t_{resolve}}^i$ du modèle intermédiaire enrichi $m_{t_{resolve}}$ sera de
type un sous-type de l'élément ciblé $e_t^i$ du modèle cible $m_t$. Les
éléments $e_{t_{resolve}}^i$ sont les éléments $e_t^i$ décorés d'une
information sur le nom de l'élément cible représenté ainsi que d'une
information sur l'élément source dont ils sont issus. En termes de métamodèle
(Figure~\ref{fig:mmresolve}), pour tout élément cible $e_t^i$ ---~instance d'un
élément $E_t^i$ du métamodèle cible $MM_t$~--- issu d'un élément source $e_s^j$
---~instance du métamodèle source $MM_s$~--- et nécessitant un élément
\emph{resolve} $e_{t_{resolve}}^i$ durant la transformation, un élément
$E_{t_{resolve}}^i$ est créé dans le métamodèle étendu $MM_{t_{resolve}}$. Cet
élément hérite de l'élément cible $E_t^i$.\\

\begin{figure}[h]
  \begin{center}
    \begin{tikzpicture}[node distance=1.1cm,>=stealth',bend angle=45,auto,scale=1,transform shape]
  \tikzstyle{every label}=[black]
  \begin{scope}

    \node (MMS) {$MM_s$};

    \node (MMSA) [right of=MMS] {} ;
    \node (MMSC) [right of=MMS,xshift=-0.2cm] {} ;

    %\node (MMRes) [right of=MMS,xshift=1cm] {$MM_{t_{resolve}}$};
    \node (MMRes) [right of=MMSA] {$MM_{t_{resolve}}$};
    
    \node (MMTA) [right of=MMRes] {} ;
    \node (MMTC) [right of=MMRes,xshift=0.2cm] {} ;
    
    %\node (MMT) [right of=MMRes,xshift=1cm] {$MM_t$};
    \node (MMT) [right of=MMTA] {$MM_t$};
    
    \node[draw,rectangle,inner sep=0.1cm] (S) [below of=MMS] {$E_s^j$};
    \node[draw,rectangle,inner sep=0.1cm] (TRes) [below of=MMRes] {$E_t^i$};
    %\node[draw,rectangle,inner sep=0.1cm] (TResRes) [below of=TRes] {Resolve$E_s^jE_t^i$};
    \node[draw,rectangle,inner sep=0.1cm] (TResRes) [below of=TRes] {$E_{t_{resolve}}^i$};
    \path (TResRes) edge [post] (TRes);
    \node[draw,rectangle,inner sep=0.1cm] (T) [below of=MMT] {$E_t^i$};

    \node (MMSB) [left of=TResRes] {} ;
    \node (MMSD) [left of=TResRes,xshift=-0.2cm] {} ;
    \node (MMTB) [right of=TResRes] {} ;
    \node (MMTD) [right of=TResRes,xshift=0.2cm] {} ;

    \path (MMSA) edge [dash pattern=on 2pt off 2pt] (MMSB);
    \path (MMTA) edge [dash pattern=on 2pt off 2pt] (MMTB);
    %\path (MMSC) edge [dash pattern=on 2pt off 2pt] (MMSD);
    %\path (MMTC) edge [dash pattern=on 2pt off 2pt] (MMTD);

  \end{scope}
\end{tikzpicture}

    \caption{Schéma d'extension du métamodèle cible par l'ajout d'éléments intermédiaires \emph{resolve}.}
    \label{fig:mmresolve}
  \end{center}
\end{figure}
Cette première phase produit donc un modèle cible non conforme au métamodèle
cible $MM_t$ de la transformation globale, mais conforme au métamodèle cible
étendu $MM_{t_{resolve}}$. Elle peut s'écrire sous la forme d'une fonction $c: MM_s
\rightarrow MM_{t_{resolve}}$ qui crée des éléments cible à partir des éléments
du modèle source.

%\todo{[ici faut illustrer avec un cas concret tiré de l'exemple, comme pour
%l'exemple TSI ; Figure~\ref{fig:approachSimpleRulesResolve}]}\\

La Figure~\ref{fig:approachSimpleRulesResolve} permet d'illustrer le mécanisme
des éléments \emph{resolve}. Nous reprenons la
Figure~\ref{fig:approachSimpleRules} décrivant les trois \emph{définitions}
composant notre transformation exemple, et nous intégrons les \emph{éléments
resolve} (représentés par des formes en pointillés). Précédemment, nous avons
vu que nous pouvions appliquer la \emph{définition} (a) complètement
indépendamment étant donné qu'elle ne nécessite aucun résultat ou partie de
résultat issu d'une autre \emph{définition}. Cette étape ne change donc pas et
ne crée aucun \emph{élément résolve}. La \emph{définition} (b) nécessite en
revanche un cercle rouge -- normalement créé dans la \emph{définition} (a) --
pour pouvoir créer un arc. Un élément dont le type (\emph{cercle pointillé})
est sous-type de l'élément cible (\emph{cercle}) est donc créé. Nous serons
donc par la suite en mesure de manipuler le \emph{cercle pointillé} comme un
\emph{cercle} classique et de filtrer sur tous les cercles, en pointillés ou
non. La couleur donnée à un \emph{élément resolve} dans la
Figure~\ref{fig:approachSimpleRulesResolve} permet de représenter l'information
de la \emph{définition} d'origine de l'élément ciblé par l'\emph{élément
resolve} et donc d'encoder le lien entre les deux éléments. Le même principe
est appliqué à la \emph{définition} (c) qui nécessite un élément normalement
créé dans la définition (b), d'où la génération d'un élément de type
\emph{carré pointillé} vert.

\begin{figure}[h]
  \centering
\begin{tabular}{c|c|c}
  %\begin{subfigure}[A]{0.30\textwidth}
  \begin{subfigure}{0.30\textwidth}
    \centering
    \definecolor{myred}{HTML}{d01e1e}
\begin{tikzpicture}[>=latex, node distance=1cm, on grid, auto]%, scale=0.6, transform shape]

\node (A2) [draw, circle, minimum size=0.5mm, color=myred] {};
\path (A2.west)+(-0.25,-1) node (A1) [draw, regular polygon, regular
polygon sides=6, minimum size=1cm, color=myred] {}; %{A};
\path[-,color=myred] (A1) edge (A2);

\path (A2.west)+(-1.5,-0.5) node (arrow) [minimum size=0.5cm] {$\longrightarrow$};
\path (arrow.west)+(-0.5,0) node (A) {A};

\end{tikzpicture}

    \subcaption{}
  \end{subfigure}
  &
  %\begin{subfigure}[seq]{0.30\textwidth}
  \begin{subfigure}{0.30\textwidth}
    \centering
    \definecolor{mygreen}{HTML}{129d1c}
\definecolor{myred}{HTML}{d01e1e}
\begin{tikzpicture}[>=latex, node distance=1cm, on grid, auto]

\node (S1) [draw, regular polygon, regular polygon sides=3, minimum
size=1cm, shape border rotate=180, color=mygreen] {}; %{;};
\path (S1.west)+(-0.5,-1) node (A2) [draw, circle, minimum size=0.5mm, dashed, color=myred] {};
\path (S1.east)+(0.5,-1) node (S2) [draw, rectangle, minimum size=0.5mm, color=mygreen] {};

\path[-,color=mygreen] (A2) edge (S1);
\path[-,color=mygreen] (S1) edge (S2);

\path (S1.west)+(-1,-0.25) node (arrow) [minimum size=0.5cm] {$\longrightarrow$};
\path (arrow.west)+(-0.5,0) node (seq) {;};

\end{tikzpicture}

    \subcaption{}
  \end{subfigure}
  &
  %\begin{subfigure}[B]{0.30\textwidth}
  \begin{subfigure}{0.30\textwidth}
    \centering
    \definecolor{myblue}{HTML}{0000FF}
\definecolor{mygreen}{HTML}{129d1c}
\begin{tikzpicture}[>=latex, node distance=1cm, on grid, auto]%, scale=0.6, transform shape]

\node (S2) [draw, rectangle, minimum size=0.5mm, color=mygreen, dashed] {};
\path (S2.east)+(0.25,-1) node (B1) [draw, regular polygon, regular polygon
sides=5, minimum size=1cm, color=myblue] {}; %{B};

\path[-,color=myblue] (S2) edge (B1);

\path (S2.west)+(-1,-0.5) node (arrow) [minimum size=0.5cm] {$\longrightarrow$};
\path (arrow.west)+(-0.5,0) node (B) {B};

\end{tikzpicture}

    \subcaption{}
  \end{subfigure}
\end{tabular}
  \caption{Règles de transformation de \texttt{A}, \texttt{;} et \texttt{B}
effectives avec la construction d'\emph{éléments resolve} (en pointillés
colorés).}
  \label{fig:approachSimpleRulesResolve}
\end{figure}

%\begin{figure}[h]
%  \begin{center}
%    \begin{tikzpicture}[node distance=1.3cm,>=stealth',bend
  angle=45,auto,scale=0.6,transform shape]

  \tikzstyle{place}=[circle,thick,draw=red!75,fill=red!20,minimum size=5mm]
  \tikzstyle{transition}=[rectangle,thick,draw=blue!75,
  			  fill=blue!20,minimum size=4mm]
  \tikzstyle{every label}=[black]

  \begin{scope}
    % Petri net A
  
    \node [place] (p1) [tokens=1] [xshift=-5cm]{};
    \node at (p1.north) [above, inner sep=3mm] {\textbf{WD$_{A}$}};
    \node at (p1.west) [left] {{$p_{ready}$}};

    \node [transition] (tp1) [right of=p1,dash pattern=on 2pt off 2pt]  {}
      edge [post,bend right,dash pattern=on 2pt off 2pt] (p1)
    ;
    \node at (tp1.south) [below] {{$parent_{start}$}};

    \node [transition] (t1) [below of=p1] {}
      edge [pre]  (p1)
    ;
    \node at (t1.west) [left] {{$t_{start}$}};

    %%in order to center tstart transition
    \node [place] (p) [below of=t1,circle,draw=white,fill=white] {};

    \node [place] (p2) [left of=p] {}
      edge [pre] (t1)
    ;
    \node at (p2.west) [left] {{$p_{running}$}};
    \node [place] (p3)  [right of=p] {}
      edge [pre] (t1)
    ;
    \node at (p3.west) [left] {{$p_{started}$}};

    \node [transition] (t2) [below of=p2] {}
      edge [pre]  (p2)
    ;
    \node at (t2.west) [left] {{$t_{finish}$}};

    \node [place] (p4) [below of=t2] {}
      edge [pre] (t2)
    ;
    \node at (p4.west) [left] {{$p_{finished}$}};
    
    \node [transition] (tp2) [right of=p4,dash pattern=on 2pt off 2pt] [xshift=0.5cm] {}
      edge [pre,bend left,dash pattern=on 2pt off 2pt] (p4)
    ;
    \node at (tp2.north) [above] {{$parent_{finish}$}};

%%%%%%%%%%%  
    \node [place] (tmp) [right of=tp1,circle,draw=white,fill=white] [xshift=0cm]{};
    % Petri net  Process
    %\node [place] (ppc1) [tokens=1,label=left:{$p_{ready}$}] [xshift=0cm]{}
    \node [place] (ppc1) [right of=tmp] [xshift=0cm]{};
    \node at (ppc1.north) [above, inner sep=3mm] {\textbf{P$_{root}$}};
    \node at (ppc1.east) [right] {{$p_{ready}$}};
    
    \node [transition] (tpc1) [below of=ppc1] {}
      edge [pre]  (ppc1)
    ;
    \node at (tpc1.east) [right] {{$t_{start}$}};

    \node [place] (ppc2) [below of=tpc1] {}
      edge [pre] (tpc1)
    ;
    \node at (ppc2.east) [right] {{$p_{running}$}};

    \node [transition] (tpc2) [below of=ppc2] {}
      edge [pre]  (ppc2)
%%      edge [pre,bend right,green!50!black] (pc4)
    ;
    \node at (tpc2.east) [right] {{$t_{finish}$}};

    \node [place] (ppc3) [below of=tpc2] {}
      edge [pre] (tpc2)
    ;
    \node at (ppc3.east) [right] {{$p_{finished}$}};
    
  \end{scope}

\end{tikzpicture}

%  \end{center}
%  \caption{Illustration du mécanisme \emph{resolve} avec les réseaux de Petri images d'une \texttt{WorkDefinition} et d'un \texttt{Process}.}
%  \label{fig:resolveEx}
%\end{figure} 

Durant cette première phase de transformation par parties, chaque
\emph{définition} a produit un ensemble d'éléments tous disjoints, les éléments
censés provenir d'autres \emph{définitions} ayant été représentés par des
éléments \emph{resolve}. %plusieurs de fonction {n,a} -> {n,a}

Nous encodons les \emph{définitions} par des stratégies de réécriture que nous
composons avec d'autres stratégies. Il en résulte une stratégie plus complexe
qui encode cette première phase de transformation.  Dans notre exemple de
transformation \emph{Text2Picture}, nous avons identifié trois
\emph{définitions} ---~(a), (b) et (c)~--- qui seront encodées respectivement
par les stratégies $S_a$, $S_b$ et $S_c$. Nous les combinons avec la séquence
et une stratégie de parcours ---~\emph{TopDown} ici~--- pour former la
stratégie $S_{phase1} = TopDown(Seq(S_a,S_b,S_c))$.




\subsection{Résolution - réconciliation}
%\todo{[on arrive à la phase \emph{resolve}]}
\label{approach:subsec:reconciliation}

L'application des transformations élémentaires sur le modèle source a produit
un résultat intermédiaire non conforme au métamodèle cible $MM_t$, car composé
de plusieurs résultats partiels incluant des \emph{éléments resolve}, comme
illustré par la Figure~\ref{fig:approachIntermediateResult}.

\begin{figure}[h]
  \begin{center}
   \definecolor{myred}{HTML}{d01e1e}
\definecolor{myblue}{HTML}{0000FF}
\definecolor{mygreen}{HTML}{129d1c}

\begin{tikzpicture}[>=latex, node distance=1cm, on grid, auto]
%, scale=0.6, transform shape]

%Seq
\node (S1) [draw, regular polygon, regular polygon sides=3, minimum size=1cm, shape border rotate=180, color=mygreen] {}; %{;};
\path (S1.west)+(-0.5,-1) node (SA2) [draw, circle, minimum size=0.5mm, dashed, color=myred] {};
\path (S1.east)+(0.5,-1) node (S2) [draw, rectangle, minimum size=0.5mm, color=mygreen] {};
\path[-,color=mygreen] (SA2) edge (S1);
\path[-,color=mygreen] (S1) edge (S2);

%A
\node (A1) [draw, regular polygon, regular polygon sides=6, minimum size=1cm, color=myred, left of=SA2,xshift=-0.5cm] {}; %{A};
\path (A1.east)+(0.25,1) node (A2) [draw, circle, minimum size=0.5mm, color=myred] {};
\path[-,color=myred] (A1) edge (A2);

%B
\node (B1) [draw, regular polygon, regular polygon sides=5, minimum size=1cm, color=myblue, right of=S2,xshift=0.5cm] {}; %{B};
\path (B1.west)+(-0.25,1) node (B2) [draw, rectangle, minimum size=0.5mm, color=mygreen, dashed] {};
\path[-,color=myblue] (B2) edge (B1);

\path (A2.west)+(-1.75,-0.5) node (arrow) [minimum size=0.5cm] {$\longrightarrow$};
\node at (arrow.north) (c) {$c$};
\path (arrow.west)+(-0.5,0) node (A) {A;B};

\end{tikzpicture}

   \caption{Résultat intermédiaire de la transformation, avant phase de \emph{résolution}.}
    \label{fig:approachIntermediateResult}
  \end{center}
\end{figure}


Pour obtenir un résultat cohérent conforme au métamodèle cible, il est
nécessaire d'effectuer un traitement sur ce résultat intermédiaire. C'est ce
que nous appelons la phase de \emph{résolution} ou \emph{réconciliation}, dont
le but est de fusionner des éléments disjoints pour les rendre identiques. Elle
consiste à parcourir le terme résultant, à trouver les éléments temporaires
\emph{resolve}, puis à reconstruire un terme résultat en remplaçant ces termes
temporaires par les termes qu'ils étaient censés remplacer. Étant donné que
toutes les \emph{définitions} ont été appliquées, nous sommes certains que
l'élément final existe, et qu'il peut se substituer à l'élément temporaire
examiné.

%\ttodo{mettre le schéma de la phase de résolution :
%Figure~\ref{fig:approachResolutionPhase} ou
%Figure~\ref{fig:approachResolutionPhase2} ?}
%\begin{figure}[h]
%  \begin{center}
%   \definecolor{myred}{HTML}{d01e1e}
\definecolor{myblue}{HTML}{0000FF}
\definecolor{mygreen}{HTML}{129d1c}

\begin{tikzpicture}[>=latex, node distance=1cm, on grid, auto]
%, scale=0.6, transform shape]

%Seq
\node (S1) [draw, regular polygon, regular polygon sides=3, minimum size=1cm, shape border rotate=180, color=mygreen] {}; %{;};
\path (S1.west)+(-0.5,-1) node (SA2) [draw, circle, minimum size=0.5mm, dashed, color=myred] {};
\path (S1.east)+(0.5,-1) node (S2) [draw, rectangle, minimum size=0.5mm, color=mygreen] {};
\path[-,color=mygreen] (SA2) edge (S1);
\path[-,color=mygreen] (S1) edge (S2);

%A
\node (A1) [draw, regular polygon, regular polygon sides=6, minimum size=1cm, color=myred, left of=SA2,xshift=-0.5cm] {}; %{A};
\path (A1.east)+(0.25,1) node (A2) [draw, circle, minimum size=0.5mm, color=myred] {};
\path[-,color=myred] (A1) edge (A2);

%B
\node (B1) [draw, regular polygon, regular polygon sides=5, minimum size=1cm, color=myblue, right of=S2,xshift=0.5cm] {}; %{B};
\path (B1.west)+(-0.25,1) node (B2) [draw, rectangle, minimum size=0.5mm, color=mygreen, dashed] {};
\path[-,color=myblue] (B2) edge (B1);

\path[latex'-latex',double] (A2) edge [bend left,thick] (SA2);
\path[latex'-latex',double] (S2) edge [bend left,thick] (B2);

\end{tikzpicture}

%   \caption{Phase de \emph{résolution}.}
%    \label{fig:approachResolutionPhase}
%  \end{center}
%\end{figure}

\begin{figure}[h]
  \begin{center}
   \definecolor{myred}{HTML}{d01e1e}
\definecolor{myblue}{HTML}{0000FF}
\definecolor{mygreen}{HTML}{129d1c}

\begin{tikzpicture}[>=latex, node distance=1cm, on grid, auto]
%, scale=0.6, transform shape]

%Seq
\node (S1) [draw, regular polygon, regular polygon sides=3, minimum size=1cm, shape border rotate=180, color=mygreen] {}; %{;};
\path (S1.west)+(-0.5,-1) node (SA2) [draw, circle, minimum size=0.5mm, dashed, color=myred] {};
\path (S1.east)+(0.5,-1) node (S2) [draw, rectangle, minimum size=0.5mm, color=mygreen] {};
\path[-,color=mygreen] (SA2) edge (S1);
\path[-,color=mygreen] (S1) edge (S2);

%A
\node (A1) [draw, regular polygon, regular polygon sides=6, minimum size=1cm, color=myred, left of=SA2,xshift=-0.5cm] {}; %{A};
\path (A1.east)+(0.25,1) node (A2) [draw, circle, minimum size=0.5mm, color=myred] {};
\path[-,color=myred] (A1) edge (A2);

%B
\node (B1) [draw, regular polygon, regular polygon sides=5, minimum size=1cm, color=myblue, right of=S2,xshift=0.5cm] {}; %{B};
\path (B1.west)+(-0.25,1) node (B2) [draw, rectangle, minimum size=0.5mm, color=mygreen, dashed] {};
\path[-,color=myblue] (B2) edge (B1);

\path (B2.east)+(1.75,-0.5) node (arrow) [minimum size=0.5cm] {$\longrightarrow$};
\node at (arrow.north) (r) {$r$};

%\path (arrow.west)+(-0.5,0) node (A) {A;B};
%result

    \node (rS1) [draw, regular polygon, regular polygon sides=3, minimum
    size=1cm, shape border rotate=180, color=mygreen, right of=r,
  xshift=1.5cm,yshift=0.75cm] {}; %{;};

    \path (rS1.west)+(-0.5,-1) node (rA2) [draw, circle, minimum size=0.5mm, color=myred] {};
    \path (rA2.west)+(-0.25,-1) node (rA1) [draw, regular polygon, regular
    polygon sides=6, minimum size=1cm, color=myred] {}; %{A};

  \path (rS1.east)+(0.5,-1) node (rS2) [draw, rectangle, minimum size=0.5mm, color=mygreen] {};
    \path (rS2.east)+(0.25,-1) node (rB1) [draw, regular polygon, regular polygon
    sides=5, minimum size=1cm, color=myblue] {}; %{B};

    \path[-,color=myred] (rA1) edge (rA2);
    \path[-,color=mygreen] (rA2) edge (rS1);
    \path[-,color=mygreen] (rS1) edge (rS2);
    \path[-,color=myblue] (rS2) edge (rB1);

\end{tikzpicture}

   \caption{Phase de \emph{résolution}.}
    \label{fig:approachResolutionPhase2}
  \end{center}
\end{figure}

Cette phase de résolution est elle-même encodée par une stratégie $S_{phase2}$
qui réécrit un terme-\emph{resolve} (c'est-à-dire la représentation d'un modèle
contenant des éléments intermédiaires \emph{resolve} sous la forme d'un terme)
en terme cible, conforme à la signature cible.

Ce remplacement est possible grâce aux informations supplémentaires qui
enrichissent le type cible ainsi qu'aux informations que nous sauvegardons
durant la transformation. Ces informations additionnelles sont des informations
sur les relations existant entre les éléments cibles et les éléments sources
dont ils sont issus. Elles sont obtenues par le biais d'actions explicites de
la part de l'utilisateur : tout terme créé dans une transformation peut être
tracé sur demande. C'est ce qu'illustre la
Figure~\ref{fig:approachSimpleRulesTrace} : les jetons colorés correspondent à
une action explicite de trace d'un terme qui a été créé dans la
\emph{définition}. Ainsi, dans cet exemple, l'utilisateur a marqué un élément
dans la \emph{définition} (a) qui correspond à l'élément ciblé par
l'\emph{élément resolve} de la définition (b). Il en est de même avec l'élément
de type \emph{carré} des \emph{définitions} (b) et (c).

\begin{figure}[h]
  \centering
\begin{tabular}{c|c|c}
  %\begin{subfigure}[A]{0.30\textwidth}
  \begin{subfigure}{0.30\textwidth}
    \centering
    \definecolor{myred}{HTML}{d01e1e}
\definecolor{mygreen}{HTML}{129d1c}
\begin{tikzpicture}[>=latex, node distance=1cm, on grid, auto, every token/.style={color=mygreen}]%, scale=0.6, transform shape]

\node (A2) [draw, circle, minimum size=0.5mm, color=myred, tokens=1] {};
\path (A2.west)+(-0.25,-1) node (A1) [draw, regular polygon, regular
polygon sides=6, minimum size=1cm, color=myred] {}; %{A};
\path[-,color=myred] (A1) edge (A2);

\path (A2.west)+(-1.5,-0.5) node (arrow) [minimum size=0.5cm] {$\longrightarrow$};
\path (arrow.west)+(-0.5,0) node (A) {A};

\end{tikzpicture}

    \subcaption{}
  \end{subfigure}
  &
  %\begin{subfigure}[seq]{0.30\textwidth}
  \begin{subfigure}{0.30\textwidth}
    \centering
    \definecolor{mygreen}{HTML}{129d1c}
\definecolor{myred}{HTML}{d01e1e}
\definecolor{myblue}{HTML}{0000FF}
\begin{tikzpicture}[>=latex, node distance=1cm, on grid, auto, every token/.style={color=myblue}]

\node (S1) [draw, regular polygon, regular polygon sides=3, minimum
size=1cm, shape border rotate=180, color=mygreen] {}; %{;};
\path (S1.west)+(-0.5,-1) node (A2) [draw, circle, minimum size=0.5mm, dashed, color=myred] {};
\path (S1.east)+(0.5,-1) node (S2) [draw, rectangle, minimum size=0.5mm, color=mygreen, tokens=1] {};

\path[-,color=mygreen] (A2) edge (S1);
\path[-,color=mygreen] (S1) edge (S2);

\path (S1.west)+(-1,-0.25) node (arrow) [minimum size=0.5cm] {$\longrightarrow$};
\path (arrow.west)+(-0.5,0) node (seq) {;};

\end{tikzpicture}

    \subcaption{}
  \end{subfigure}
  &
  %\begin{subfigure}[B]{0.30\textwidth}
  \begin{subfigure}{0.30\textwidth}
    \centering
    \definecolor{myblue}{HTML}{0000FF}
\definecolor{mygreen}{HTML}{129d1c}
\begin{tikzpicture}[>=latex, node distance=1cm, on grid, auto]%, scale=0.6, transform shape]

\node (S2) [draw, rectangle, minimum size=0.5mm, color=mygreen, dashed] {};
\path (S2.east)+(0.25,-1) node (B1) [draw, regular polygon, regular polygon
sides=5, minimum size=1cm, color=myblue] {}; %{B};

\path[-,color=myblue] (S2) edge (B1);

\path (S2.west)+(-1,-0.5) node (arrow) [minimum size=0.5cm] {$\longrightarrow$};
\path (arrow.west)+(-0.5,0) node (B) {B};

\end{tikzpicture}

    \subcaption{}
  \end{subfigure}
\end{tabular}
  \caption{Règles de transformation de \texttt{A}, \texttt{;} et \texttt{B}
effectives, avec traçage des éléments correspondant à un \emph{élément resolve}
d'une autre \emph{définition} (token coloré de la couleur du résultat d'une
\emph{définition)}.}
  \label{fig:approachSimpleRulesTrace}
\end{figure}

Une autre approche possible eût été de tracer systématiquement tous les termes
créés, mais nous avons fait ce choix dans le but d'améliorer la lisibilité de
la trace. %\ttodo{revoir tout ça, ça va changer} 
Toutes ces informations
supplémentaires constituent ce que nous appelons le modèle de lien. Il
maintient tout au long de la transformation des relations entre les éléments
sources et les éléments cibles qui en sont issus. 
Dans ce cas précis, le traçage des liens a eu un usage purement mécanique pour
permettre la résolution de liens. Cependant, outre cet usage pour la phase de
résolution, le construction dédiée au marquage de termes et le modèle de lien
nous permettent d'assurer la traçabilité de la transformation à des fins de
vérification \emph{a posteriori}. Nous traitons de ce sujet dans le
chapitre~\ref{ch:traceability}.

Ainsi, une transformation $T$ est la composée de deux fonctions $c : MM_s
\rightarrow MM_{t_{resolve}}$ et $r : MM_{t_{resolve}} \rightarrow MM_t$. La
transformation complète $T : MM_s \rightarrow MM_t$ est donc définie par $T = r
\circ c$. L'encodage d'une telle fonction dans notre approche est une stratégie
$S$ mettant en séquence les deux stratégies représentant chaque phase,  qui
peut se résumer par $S = Seq(S_{phase1}, S_{phase2})$.

Outre les avantages évoqués en début de chapitre, un des intérêts de cette
approche compositionnelle reposant sur les stratégies de réécriture apparait
immédiatement : reposant sur les stratégies de réécriture, nous bénéficions
naturellement de la modularité intrinsèque à ce concept que le langage de
stratégies de {\tom} implémente. Il est ainsi possible de réutiliser les
\emph{définitions} dans une autre transformation, sans adaptation lourde du
code. On pourrait imaginer que la phase de résolution devienne bloquante dans
ce cas, cependant, comme nous le verrons dans la description de
l'implémentation de notre approche (chapitre~\ref{ch:outils}), cette phase est
générée et peut aussi être générée sous la forme de plusieurs stratégies
distinctes ---~plus simples~--- réutilisables dans le cadre d'une autre
transformation.


%\begin{figure}[h]
%  \begin{center}
%    \begin{tikzpicture}[node distance=1.1cm,>=stealth',bend angle=45,auto,scale=1,transform shape]
  \tikzstyle{every label}=[black]
  \begin{scope}

    \node (MMS) {$MM_{Text}$};

    \node (MMSA) [right of=MMS,xshift=0.5cm] {} ;
    \node (MMSC) [right of=MMS,xshift=-0.2cm] {} ;

    %\node (MMRes) [right of=MMS,xshift=1cm] {$MM_{t_{resolve}}$};
    \node (MMRes) [right of=MMSA,xshift=0.5cm] {$MM_{Picture_{resolve}}$};
    
    \node (MMTA) [right of=MMRes,xshift=0.5cm] {} ;
    \node (MMTC) [right of=MMRes,xshift=0.2cm] {} ;
    
    %\node (MMT) [right of=MMRes,xshift=1cm] {$MM_t$};
    \node (MMT) [right of=MMTA,xshift=0.5cm] {$MM_{Picture}$};
    
    \node[draw,rectangle,inner sep=0.1cm] (S) [below of=MMS] {$A$};
    \node[draw,rectangle,inner sep=0.1cm] (TRes) [below of=MMRes] {$Circle$};
    %\node[draw,rectangle,inner sep=0.1cm] (TResRes) [below of=TRes] {Resolve$E_s^jE_t^i$};
    \node[draw,rectangle,inner sep=0.1cm] (TResRes) [below of=TRes] {$ResolveACircle$};
    \path (TResRes) edge [post] (TRes);
    \node[draw,rectangle,inner sep=0.1cm] (T) [below of=MMT] {$Circle$};

    \node (MMSB) [left of=TResRes,xshift=-0.5cm] {} ;
    \node (MMSD) [left of=TResRes,xshift=-0.2cm] {} ;
    \node (MMTB) [right of=TResRes,xshift=0.5cm] {} ;
    \node (MMTD) [right of=TResRes,xshift=0.2cm] {} ;

    \path (MMSA) edge [dash pattern=on 2pt off 2pt] (MMSB);
    \path (MMTA) edge [dash pattern=on 2pt off 2pt] (MMTB);
    %\path (MMSC) edge [dash pattern=on 2pt off 2pt] (MMSD);
    %\path (MMTC) edge [dash pattern=on 2pt off 2pt] (MMTD);

  \end{scope}
\end{tikzpicture}

%    \caption{Instanciation du schéma d'extension du métamodèle cible pour
%    l'exemple Text2Picture.}
%    \label{fig:mmresolveinst}
%  \end{center}
%\end{figure}




%\section{\todo{ici ? }Travail connexe}
%\todo{QVT, ATL, scheduling, réconciliation de références, }
%
%Cette approche 




%\section{\todo{FROM TSI}}
%
%\todo{Note : Reprendre \textbf{en gros} la section "approche" de TSI. MAIS,
%séparer en deux parties, et dissoudre la formalisation dans l'explication.}
%
%
%\subsection{Formalisation de l'approche \ttodo{à dissoudre}}
%\label{sec:formalisation}
%formalisation ?

%Dans notre approche, une transformation $T$ est donc constituée de deux
%phases distinctes qui peuvent être vues comme des fonctions. La première
%phase $c: MM_s \rightarrow MM_{t_{resolve}}$ consiste à créer des
%éléments cible à partir des éléments du modèle source. Des
%éléments \emph{resolve} étant créés et intégrés au résultat
%durant cette phase, le modèle résultant est conforme au métamodèle
%cible étendu, noté $MM_{t_{resolve}}$.\\
%Cette extension du métamodèle cible passe par un enrichissement du type
%cible (ajout d'informations additionnelles). Ainsi, tout élément
%\emph{resolve} $e_{t_{resolve}}^i$ du modèle intermédiaire enrichi
%$m_{t_{resolve}}$ sera de type un sous-type de l'élément associé $e_t^i$
%du modèle cible $m_t$. Les éléments $e_{t_{resolve}}^i$ sont les
%éléments $e_t^i$ décorés d'une information sur le nom de l'élément
%cible représenté ainsi que d'une information sur l'élément source dont
%ils sont issus. En termes de métamodèle (Figure~\ref{fig:mmresolve}), pour
%tout élément cible $e_t^i$ -- instance d'un élément $E_t^i$ du
%métamodèle cible $MM_t$ -- issu d'un élément source $e_s^j$ -- instance
%du métamodèle source $MM_s$ -- et nécessitant un élément
%\emph{resolve} $e_{t_{resolve}}^i$ durant la transformation, un élément
%$E_{t_{resolve}}^i$ est créé dans le métamodèle étendu
%$MM_{t_{resolve}}$. Cet élément hérite de l'élément cible $E_t^i$.
%
%\begin{figure}[h]
%  \begin{center}
%    \begin{tikzpicture}[node distance=1.1cm,>=stealth',bend angle=45,auto,scale=1,transform shape]
  \tikzstyle{every label}=[black]
  \begin{scope}

    \node (MMS) {$MM_s$};

    \node (MMSA) [right of=MMS] {} ;
    \node (MMSC) [right of=MMS,xshift=-0.2cm] {} ;

    %\node (MMRes) [right of=MMS,xshift=1cm] {$MM_{t_{resolve}}$};
    \node (MMRes) [right of=MMSA] {$MM_{t_{resolve}}$};
    
    \node (MMTA) [right of=MMRes] {} ;
    \node (MMTC) [right of=MMRes,xshift=0.2cm] {} ;
    
    %\node (MMT) [right of=MMRes,xshift=1cm] {$MM_t$};
    \node (MMT) [right of=MMTA] {$MM_t$};
    
    \node[draw,rectangle,inner sep=0.1cm] (S) [below of=MMS] {$E_s^j$};
    \node[draw,rectangle,inner sep=0.1cm] (TRes) [below of=MMRes] {$E_t^i$};
    %\node[draw,rectangle,inner sep=0.1cm] (TResRes) [below of=TRes] {Resolve$E_s^jE_t^i$};
    \node[draw,rectangle,inner sep=0.1cm] (TResRes) [below of=TRes] {$E_{t_{resolve}}^i$};
    \path (TResRes) edge [post] (TRes);
    \node[draw,rectangle,inner sep=0.1cm] (T) [below of=MMT] {$E_t^i$};

    \node (MMSB) [left of=TResRes] {} ;
    \node (MMSD) [left of=TResRes,xshift=-0.2cm] {} ;
    \node (MMTB) [right of=TResRes] {} ;
    \node (MMTD) [right of=TResRes,xshift=0.2cm] {} ;

    \path (MMSA) edge [dash pattern=on 2pt off 2pt] (MMSB);
    \path (MMTA) edge [dash pattern=on 2pt off 2pt] (MMTB);
    %\path (MMSC) edge [dash pattern=on 2pt off 2pt] (MMSD);
    %\path (MMTC) edge [dash pattern=on 2pt off 2pt] (MMTD);

  \end{scope}
\end{tikzpicture}

%    \caption{Schéma d'extension du métamodèle cible par l'ajout d'éléments intermédiaires \emph{resolve}.}
%    \label{fig:mmresolve}
%  \end{center}
%\end{figure}
%
%\noindent La seconde phase $r : MM_{t_{resolve}} \rightarrow MM_t$ consiste à
%éliminer ces éléments intermédiaires.\\
%La transformation complète $T : MM_s \rightarrow MM_t$ est
%définie par $T = r \circ c$.
%
%\noindent En instanciant notre approche avec le cas d'étude SimplePDLToPetriNet, nous
%obtenons : %$SimplePDLToPetriNet$ comme la composée de $Transformer$ et de
%%$Resolve$, avec $MM_{SimplePDL}$, $MM_{PetriNet_{resolve}}$ et $MM_{PetriNet}$
%%respectivement les métamodèles source, étendu et cible :
%\begin{tabbing}
%  $SimplePDLToPetriNet$ \= $ : MM_{SimplePDL} \rightarrow MM_{PetriNet}$\\
%  $Transformer$ \> $ : MM_{SimplePDL} \rightarrow MM_{PetriNet_{resolve}}$\\
%  $Resolve$ \> $ : MM_{PetriNet_{resolve}} \rightarrow MM_{PetriNet}$\\
%  $SimplePDLToPetriNet $ \> $ = Resolve \circ Transformer$\\
%\end{tabbing}
%Appliquée au processus $p_{root}$ (illustré Figure~\ref{fig:simplepdlusecase})
%conforme à $MM_{SimplePDL}$ (Figure~\ref{fig:simplepdlmmodel}), cette
%transformation produira un réseau de Petri $pn$ (illustré
%Figure~\ref{fig:petrinetusecase}) conforme à $MM_{PetriNet}$
%(Figure~\ref{fig:petrinetmmodel}), en passant par le résultat intermédiaire
%$pn_{resolve}$ conforme au métamodèle $MM_{PetriNet_{resolve}}$. On obtient
%donc :
%\begin{flushleft}
%  $SimplePDLToPetriNet(p_{root}) = Resolve(Transformer(p_{root}))$, avec\\
%  $Transformer(p_{root}) = pn_{resolve}$ et\\ 
%  $Resolve(p_{resolve}) = pn$
%\end{flushleft}
%La Figure~\ref{fig:mmresolveinst} instancie le schéma d'extension du
%métamodèle cible au cas d'étude : le métamodèle cible est enrichi
%d'une métaclasse \emph{ResolvePT} pour pouvoir créer des éléments
%intermédiaires \emph{resolve} qui jouent temporairement le role d'une
%\emph{Transition} obtenue à partir d'une source \emph{Process}.
%\begin{figure}[h]
%  \begin{center}
%    \begin{tikzpicture}[node distance=1.1cm,>=stealth',bend angle=45,auto,scale=1,transform shape]
  \tikzstyle{every label}=[black]
  \begin{scope}

    \node (MMS) {$MM_{Text}$};

    \node (MMSA) [right of=MMS,xshift=0.5cm] {} ;
    \node (MMSC) [right of=MMS,xshift=-0.2cm] {} ;

    %\node (MMRes) [right of=MMS,xshift=1cm] {$MM_{t_{resolve}}$};
    \node (MMRes) [right of=MMSA,xshift=0.5cm] {$MM_{Picture_{resolve}}$};
    
    \node (MMTA) [right of=MMRes,xshift=0.5cm] {} ;
    \node (MMTC) [right of=MMRes,xshift=0.2cm] {} ;
    
    %\node (MMT) [right of=MMRes,xshift=1cm] {$MM_t$};
    \node (MMT) [right of=MMTA,xshift=0.5cm] {$MM_{Picture}$};
    
    \node[draw,rectangle,inner sep=0.1cm] (S) [below of=MMS] {$A$};
    \node[draw,rectangle,inner sep=0.1cm] (TRes) [below of=MMRes] {$Circle$};
    %\node[draw,rectangle,inner sep=0.1cm] (TResRes) [below of=TRes] {Resolve$E_s^jE_t^i$};
    \node[draw,rectangle,inner sep=0.1cm] (TResRes) [below of=TRes] {$ResolveACircle$};
    \path (TResRes) edge [post] (TRes);
    \node[draw,rectangle,inner sep=0.1cm] (T) [below of=MMT] {$Circle$};

    \node (MMSB) [left of=TResRes,xshift=-0.5cm] {} ;
    \node (MMSD) [left of=TResRes,xshift=-0.2cm] {} ;
    \node (MMTB) [right of=TResRes,xshift=0.5cm] {} ;
    \node (MMTD) [right of=TResRes,xshift=0.2cm] {} ;

    \path (MMSA) edge [dash pattern=on 2pt off 2pt] (MMSB);
    \path (MMTA) edge [dash pattern=on 2pt off 2pt] (MMTB);
    %\path (MMSC) edge [dash pattern=on 2pt off 2pt] (MMSD);
    %\path (MMTC) edge [dash pattern=on 2pt off 2pt] (MMTD);

  \end{scope}
\end{tikzpicture}

%    \caption{Instanciation du schéma d'extension du métamodèle cible pour le cas d'étude SimplePDLToPetriNetPetriNet.}
%    \label{fig:mmresolveinst}
%  \end{center}
%\end{figure}


\section{Validation par un cas d'étude}

Pour valider la proposition, nous nous sommes appuyés sur une étude
de cas : la transformation \emph{SimplePDLToPetriNet}. Nous ne rentrons pour le
moment pas dans les détails et n'expliquons pas précisément les métamodèles des
formalismes considérés. Nous préciserons la cas dans le
chapitre~\ref{ch:traceability} qui suit, puis nous le développerons dans son
intégralité dans le chapitre~\ref{ch:usecase} avec son implémentation. Pour
résumer cette étude de cas, l'objectif est de transformer des processus
génériques décrits dans le formalisme SimplePDL en leur représentation sous la
forme de réseaux de Petri. Par exemple, la
figure~\ref{fig:simplesimplepdlprocess} décrit un processus simple nommé
\emph{root} composé de deux activités : \emph{A} et \emph{B}. Ces deux
activités sont liées par une contrainte de précédence \emph{startToFinish}
(\emph{s2f}). Cela signifie que l'activité \emph{A} doit avoir commencé pour
pouvoir terminer l'activité \emph{B}.

\begin{figure}[h] 
  \begin{center}
    %\begin{tikzpicture}[node distance=1.1cm,>=stealth',bend angle=45,auto,scale=0.75,transform shape]
\begin{tikzpicture}[node distance=1.1cm,>=stealth',bend angle=45,auto,scale=1,transform shape]
  \tikzstyle{every label}=[black]
  \begin{scope}
    \node (A) [xshift=0cm] {\textbf{A}};
    \node (B) [right of=A,xshift=0.5cm]  {\textbf{B}};
    \path (A) edge [post] node {s2f} (B);
    \node[draw,rectangle,inner sep=0.4cm,fit=(A) (B)] (proot) {};
    \node at (proot.north west) [above, inner sep=1mm,xshift=0.4cm] {\textbf{$_{root:}$}};
  \end{scope}
\end{tikzpicture}
%\caption{SimplePDL use case}
%\captionof{figure}{Instance de SimplePDL}

    \caption{Exemple de processus SimplePDL.}
    \label{fig:simplesimplepdlprocess}
  \end{center}
\end{figure}

Ce processus générique peut s'exprimer sous la forme d'un réseau de Petri tel
qu'illustré par la figure~\ref{fig:simplepetrinetprocess} suivante. Les places
sont représentées par des cercles rouges, les transitions par des carrés bleus
et les arcs par des flèches. %Les nœuds en pointillés sont des éléments
%intermédiaires \emph{resolve}. 
Les flèches en pointillés sont des arcs de synchronisation, celles en trait
plein noir sont des arcs normaux créés dans des différentes \emph{définitions},
la flèche verte est un arc obtenu par la \emph{définition} transformant la
séquence (qui impose la contrainte de précédence). Dans ce réseau de Petri,
lorsque la première transition de $P_{root}$ est franchie, le jeton de la
première place est ajouté à la seconde place de $P_{root}$. Un jeton est aussi
ajouté aux premières places des réseaux $A$ et $B$, ce qui a pour effet de
démarrer les tâches. La transition $t_{finish}$ de $B$ ne peut être franchie
que si $A$ a démarré.

\begin{figure}[h]
  \begin{center}
    %\begin{tikzpicture}[node distance=1.1cm,>=stealth',bend
%angle=45,auto,scale=0.50,transform shape]

\begin{tikzpicture}[node distance=1.1cm,>=stealth',bend
  angle=45,auto,scale=1,transform shape]

  \tikzstyle{place}=[circle,thick,draw=red!75,fill=red!20,minimum size=5mm]
  \tikzstyle{transition}=[rectangle,thick,draw=blue!75,
  			  fill=blue!20,minimum size=4mm]

  \tikzstyle{every label}=[black]

  \begin{scope}
    % Petri net root
    \node [place] (ppr1) [tokens=1]{};%,xshift=14.5cm]{};

    %\node at (ppr1.west) [inner sep=1mm,label=above:{$p_{ready}$},xshift=-2mm] {};
    \node at (ppr1.north) [above, inner sep=2.5mm] {\textbf{P$_{root}$}};

    \node [transition] (tpr1) [below of=ppr1] {}
      edge [pre]  (ppr1)
    ;

    \node [place] (ppr2) [below of=tpr1] {}
      edge [pre] (tpr1)
    ;

    \node [transition] (tpr2) [below of=ppr2] {}
      edge [pre]  (ppr2)
    ;

    \node [place] (ppr3) [below of=tpr2] {}
      edge [pre] (tpr2)
    ;

    % Petri net A
    %%\node [place] (p1) [tokens=0,label=right:{$p_{ready}$}] [xshift=-12cm]{}
    \node [place] (pa1) [tokens=0,xshift=3cm]{} %[xshift=12.2cm] {};
      edge [pre,bend right,dash pattern=on 2pt off 2pt,black!50!black] (tpr1);
    \node at (pa1.north) [above, inner sep=3mm] {\textbf{A}};

    \node [transition] (ta1) [below of=pa1] {}
      edge [pre]  (pa1)
    ;

    %%in order to center tstart transition
    \node [place] (p) [below of=ta1,circle,draw=white,fill=white] {};

    \node [place] (pa2) [left of=p] {}
      edge [pre] (ta1)
    ;
    \node [place] (pa3)  [right of=p] {}
      edge [pre] (ta1)
    ;
    \node at (pa3.west) [left] {{$p_{started}$}};

    \node [transition] (ta2) [below of=pa2] {}
      edge [pre]  (pa2)
    ;

    \node [place] (pa4) [below of=ta2] {}
      edge [post,bend left,dash pattern=on 2pt off 2pt,black!50!black] (tpr2)
      edge [pre] (ta2)
    ;


    % Petri net B
    \node [place] (pb1) [tokens=0,xshift=7cm]{}% [xshift=8.9cm]{};
      edge [pre,bend right,dash pattern=on 2pt off 2pt,black!50!black] (tpr1);
    \node at (pb1.north) [above, inner sep=3mm] {\textbf{B}};

    \node [transition] (tb1) [below of=pb1] {}
      edge [pre] (pb1);

    %%in order to center tstart transition
    \node [place] (pp) [below of=tb1,circle,draw=white,fill=white] {};
    \node [place] (pb2) [left of=pp] {}
      edge [pre] (tb1)
    ;
    \node [place] (pb3) [right of=pp] {}
      edge [pre] (tb1)
    ;

    \node [transition] (tb2) [below of=pb2] {}
      edge [pre]  (pb2) ;
    \node at (tb2.east) [right] {{$t_{finish}$}};
    \path (pa3) edge [post,bend left,green!50!black,thick] node
    [xshift=-0.5cm,yshift=-0.5cm,green!50!black]{s2f} (tb2);

    \node [place] (pb4) [below of=tb2] {}
      edge [post,bend left,dash pattern=on 2pt off 2pt,black!50!black] (tpr2)
      edge [pre] (tb2)
    ;



%\node[draw,rectangle,inner sep=0.8cm,fit=(pb4) (ppr1) (ppr3)] (proot) {};
  \end{scope}

\end{tikzpicture}
%\caption{Complete Process described in the use case}

    \caption{Réseau de Petri correspondant au processus décrit par la
      figure~\ref{fig:simplesimplepdlprocess}.}
    \label{fig:simplepetrinetprocess}
  \end{center}
\end{figure}

Cette transformation peut être décomposée en trois \emph{définitions}. Chacune
d'entre elles produit un réseau de Petri. Nous les représentons toutes les
trois dans la figure~\ref{fig:atomicpn} qui suit (les nœuds en pointillés sont
des éléments intermédiaires \emph{resolve}). 

\begin{figure}
  \begin{center}
    %\begin{figure}
%  \begin{center}
  \begin{tabular}{ccccc}
  Process & & WorkDefinition & & WorkSequence \\
    $\downarrow$ & & $\downarrow$ & & $\downarrow$ \\
%    \begin{subfigure}[h]{0.25\linewidth}
    \begingroup
\tikzset{every picture/.style={scale=0.7}}%
    \begin{tikzpicture}[node distance=1.1cm,>=stealth',bend
  angle=45,auto,transform shape]

  \tikzstyle{place}=[circle,thick,draw=red!75,fill=red!20,minimum size=5mm]
  \tikzstyle{transition}=[rectangle,thick,draw=blue!75,
  			  fill=blue!20,minimum size=4mm]

  \tikzstyle{every label}=[black]

  \begin{scope}
    % Petri net  Process
    \node [place] (ppc1) [tokens=0] [xshift=0cm]{}
    ;
    \node at (ppc1.west) [left] {};
    
%    \node [transition] (tp1) [right of=ppc1,dash pattern=on 2pt off 2pt]  {}
%      edge [post,bend right,dash pattern=on 2pt off 2pt] (ppc1)
%    ;
%    \node at (tp1.south) [below] {{$source$}};

    \node [transition] (tpc1) [below of=ppc1] {}
      edge [pre]  (ppc1)
    ;
    \node at (tpc1.west) [left] {};

    \node [place] (ppc2) [below of=tpc1] {}
      edge [pre] (tpc1)
    ;
    \node at (ppc2.west) [left] {};

    \node [transition] (tpc2) [below of=ppc2] {}
      edge [pre]  (ppc2)
    ;
    \node at (tpc2.west) [left] {};

    \node [place] (ppc3) [below of=tpc2] {}
      edge [pre] (tpc2)
    ;
    \node at (ppc3.west) [left] {};
    
%    \node [transition] (tp2) [right of=ppc3,dash pattern=on 2pt off 2pt] {}
%      edge [pre,bend left,dash pattern=on 2pt off 2pt] (ppc3)
%    ;
%    \node at (tp2.north) [above] {{$target$}};
  \end{scope}

\end{tikzpicture}
 
\endgroup
%    \end{subfigure}
    & ~~~ &
%    \begin{subfigure}[h]{0.30\linewidth}
    \begingroup
\tikzset{every picture/.style={scale=0.6}}%
    \begin{tikzpicture}[node distance=1.3cm,>=stealth',bend
  angle=45,auto,transform shape]

  \tikzstyle{place}=[circle,thick,draw=red!75,fill=red!20,minimum size=5mm]
  \tikzstyle{transition}=[rectangle,thick,draw=blue!75,
  			  fill=blue!20,minimum size=4mm]

  \tikzstyle{every label}=[black]

  \begin{scope}
    % Petri net A
    %\node [place] (p1) [tokens=1,label=left:{$p_{ready}$}] [xshift=-5cm]{}
    \node [place] (p1) [tokens=0] [xshift=-5cm]{}
    ;
    \node at (p1.west) [left] {};

    %\node [transition] (tp1) [right of=p1,dash pattern=on 2pt off 2pt,label=below:{$parent_{start}$}]  {}
    \node [transition] (tp1) [right of=p1,dash pattern=on 2pt off 2pt]  {}
      edge [post,bend right,dash pattern=on 2pt off 2pt] (p1)
    ;
    \node at (tp1.south) [below] {};

    %\node [transition] (t1) [below of=p1,label=left:{$t_{start}$}] {}
    \node [transition] (t1) [below of=p1] {}
      edge [pre]  (p1)
    ;
    \node at (t1.west) [left] {};

    %%in order to center tstart transition
    \node [place] (p) [below of=t1,circle,draw=white,fill=white] {};

    %\node [place] (p2) [left of=p,label=left:{$p_{running}$}] {}
    \node [place] (p2) [left of=p] {}
      edge [pre] (t1)
    ;
    \node at (p2.west) [left] {};
    %\node [place] (p3)  [right of=p,label=left:{$p_{started}$}] {}
    \node [place] (p3)  [right of=p] {}
      edge [pre] (t1)
    ;
    \node at (p3.west) [left] {};

    %\node [transition] (t2) [below of=p2,label=left:{$t_{finish}$}] {}
    \node [transition] (t2) [below of=p2] {}
      edge [pre]  (p2)
    ;
    \node at (t2.west) [left] {};

    %\node [place] (p4) [below of=t2,label=left:{$p_{finished}$}] {}
    \node [place] (p4) [below of=t2] {}
      edge [pre] (t2)
    ;
    \node at (p4.west) [left] {};
    
    %\node [transition] (tp2) [right of=p4,dash pattern=on 2pt off 2pt,label=above:{$parent_{finish}$}] [xshift=0.5cm] {}
    \node [transition] (tp2) [right of=p4,dash pattern=on 2pt off 2pt] [xshift=0.5cm] {}
      edge [pre,bend left,dash pattern=on 2pt off 2pt] (p4)
    ;
    \node at (tp2.north) [above] {};

  \end{scope}

\end{tikzpicture}
 
\endgroup
%    \end{subfigure}
    & ~~~ &
%    \begin{subfigure}[h]{0.35\linewidth}
    \begingroup
\tikzset{every picture/.style={scale=0.6}}%
    \begin{tikzpicture}[node distance=1.2cm,>=stealth',bend
  angle=25,auto,transform shape]

  \tikzstyle{place}=[circle,thick,draw=red!15,fill=red!5,minimum size=5mm]
  \tikzstyle{transition}=[rectangle,thick,draw=blue!15,fill=blue!5,minimum size=4mm]

  \tikzstyle{edge}=[black!25!black!25]
  \tikzstyle{every label}=[black]

  \begin{scope}
    % Petri net A
    \node [place] (p1) [tokens=0] [xshift=-3.5cm]{}
    ;

    \node [transition] (t1) [below of=p1] {}
      edge [pre,black!25]  (p1)
    ;

    %%in order to center tstart transition
    \node [place] (p) [below of=t1,circle,draw=white,fill=white] {};

    \node [place] (p2) [left of=p] {}
      edge [pre,black!25] (t1)
    ;

    \node [place] (p3)  [right of=p,draw=red!75,fill=red!20,dash pattern=on 2pt off 2pt] {}
      edge [pre,black!25] (t1)
    ;
%    \node at (p3.west) [left] {{$start$}};

    \node [transition] (t2) [below of=p2] {}
      edge [pre,black!25]  (p2)
    ;

    \node [place] (p4) [below of=t2] {}
      edge [pre,black!25] (t2)
    ;

    % Petri net B
    \node [place] (pb1) [tokens=0] {}
    ;

    %dotted,
    \node [transition] (tb1) [below of=pb1] {}
      edge [pre,black!25] (pb1)
      ;

    %%in order to center tstart transition
    \node [place] (pp) [below of=tb1,circle,draw=white,fill=white] {};
    \node [place] (pb2) [left of=pp] {}
      edge [pre,black!25] (tb1)
    ;
    \node [place] (pb3) [right of=pp] {}
      edge [pre,black!25] (tb1)
    ;

    \node [transition] (tb2) [below of=pb2,draw=blue!75,fill=blue!20,dash pattern=on 2pt off 2pt] {}
      edge [pre,black!25] (pb2)
      edge [pre,bend right,green!50!black] (p3)
      ;
%    \node at (tb2.east) [right] {{$finish$}};

    \node [place] (pb4) [below of=tb2] {}
      edge [pre,black!25] (tb2)
    ;

  \end{scope}

\end{tikzpicture}

\endgroup
%    \end{subfigure}
    \\
    %\emph{a. ProcessToPetriNet} & \emph{b. WorkDefinitionToPetriNet} & \emph{c. WorkSequenceToPetriNet}\\
    %\emph{(a)} & \emph{(b)} & \emph{(c)}\\
    \end{tabular}
%    \caption{Transformations élémentaires composant \texttt{SimplePDLToPetriNet}}
%    \label{fig:atomicpn}
%  \end{center}
%\end{figure}

    \caption{Transformations élémentaires composant \texttt{SimplePDLToPetriNet}.}
    \label{fig:atomicpn}
  \end{center}
\end{figure}

Dans cette version du cas d'étude, le mécanisme de création d'éléments
temporaires \emph{resolve} est utilisé dans le cadre de deux
\emph{définitions}, celle qui transforme les \emph{WorkDefinitions} et celle
qui transforme les \emph{WorkSequences}. Le mécanisme de marquage est quant à
lui utilisé dans la \emph{définition} qui transforme les \emph{Process} afin
d'effectuer la correspondance avec les éléments \emph{resolve} créés dans la
transformation élémentaire qui prend en entrée une \emph{WorkDefinition}. Des
éléments sont aussi tracés dans cette dernière pour assurer la résolution avec
les éléments intermédiaires produits dans la \emph{définition} qui transforme
les \emph{Worksequences}. Nous avons mis en œuvre ce mécanisme global de
résolution que nous détaillons techniquement dans le chapitre~\ref{ch:outils}.
Le mécanisme de marquage (ou traçage) est une forme de traçabilité, la
traçabilité \emph{interne} (ou \emph{technique}) qui est étroitement liée à
l'implémentation de la transformation.

%\FloatBarrier

\section{Synthèse} \label{ch:approach:synth}

Dans ce chapitre, nous avons présenté notre approche pour transformer les
modèles. Elle est hybride étant donné qu'il ne s'agit pas d'utiliser uniquement
un langage généraliste ou un langage dédié, mais d'avoir une approche
intermédiaire où nous intégrons des constructions dédiées au sein d'un langage
généraliste. Ce procédé est rendu possible par l'utilisation du langage {\tom}
présenté dans le chapitre~\ref{ch:tom} et qui repose sur le calcul de
réécriture. Une telle approche a l'avantage de pouvoir faire bénéficier
l'utilisateur du meilleur des deux mondes des langages généralistes et dédiés à
la fois. Ainsi, l'utilisateur développant une transformation en {\tomjava} aura
les constructions spécifiques aux transformations de modèles tout en conservant
l'outillage existant de {\java}.

Nous avons aussi expliqué que notre approche se base sur la réécriture de
termes. Compte tenu du fait que nous transformons des modèles, notre approche
commence par un changement d'espace technologique rendu possible grâce à un
outil de génération d'ancrages formels que nous avons développé. Il nous permet
de représenter des modèles {\ecore} sous la forme de termes, que nous pouvons
ensuite parcourir et transformer avec des stratégies de réécriture. Ces
stratégies de réécriture encodent les transformations élémentaires composant la
transformation globale, elle-même encodée par une stratégie de réécriture. Dans
un but d'accessibilité de l'approche et des outils pour les utilisateurs, nous
avons choisi de proposer une méthode permettant à l'utilisateur de ne pas avoir
à gérer l'ordonnancement des pas d'exécution. Notre solution à ce problème est
l'introduction d'éléments intermédiaires dits \emph{resolve} qui jouent le rôle
d'éléments cibles tant que ces derniers ne sont pas créés. Une transformation
selon notre approche est donc composée de deux phases distinctes : la première
où les éléments cibles et cibles intermédiaires sont créés, et la seconde qui
consiste à \emph{résoudre} les liens, c'est-à-dire à supprimer les éléments
intermédiaires et à remplacer les liens pointant vers eux par des liens vers
les éléments cibles qu'ils représentaient.


% vim:spell spelllang=fr
