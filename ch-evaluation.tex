% vim:spell spelllang=fr
\chapter{Résultats expérimentaux}
\label{ch:evaluation}

Dans ce chapitre, nous présentons des résultats expérimentaux obtenus avec nos
outils, notamment leurs performances. Nous discutons notre approche et nos
choix technologiques, leur intérêt, ainsi que les limitations et perspectives de
notre implémentation technique.
Nous avons tenté d'évaluer nos outils afin d'améliorer les points limitants et
de parfaire nos points forts par rapport à d'autres outils tels qu'{\atl} ou
d'autres outils de transformation utilisant des stratégies de réécriture.


\section{Utilisabilité}

%\ttodo{revoir, notamment faire des références à ATL ici}

Compte tenu du fait qu'un de nos objectifs était de simplifier le processus de
développement pour les utilisateurs, un premier aspect de notre évaluation
concerne l'utilisabilité de nos outils. 
Fortement dépendant du contexte et de l'utilisateur, ce critère est
difficilement mesurable et quantifiable. Nous pouvons toutefois exposer des
retours utilisateurs.
%Ce critère ne peut être évalué de
%manière complètement formelle et objective (avec des valeurs chiffrées), mais
%nous pouvons exposer des retours utilisateurs.
 
Nous avons pu tester en conditions réelles une partie de nos outils : il
s'agissait de développer une transformation en {\tomjava} dans le cadre du
projet {\quarteft} par une entreprise partenaire du projet. Le développeur
était un ingénieur en informatique maîtrisant le langage {\java} ainsi que
l'environnement {\eclipse}. Il ne connaissait pas {\tom}, ni les langages à
base de règles de réécriture, ni le concept de stratégie avant de commencer.

La prise en main du langage {\tom} sans les nouvelles constructions s'est faite
sans aucune difficulté et sans réel besoin de support de notre part hormis la
documentation officielle en ligne. Nous avons donné un support informel pour
l'usage des stratégies {\tom} afin d'accélérer son développement. L'outil
{\tomemf} a été utilisé pour générer les ancrages formels inclus dans la
transformation. Si l'usage en lui-même n'a pas posé de problème à l'ingénieur
(peu d'options différentes, documentation centralisée), nous lui avons fourni
un support plus important. En effet, l'usage par un non développeur a permis de
découvrir des \emph{bugs} et des manques de fonctionnalités (besoin de
génération massive de \emph{mappings} en une seule fois qui a donné lieu à la
fonctionnalité de génération multiple de \emph{EPackages}, conflits de noms
qui a donné lieu à la fonctionnalité de préfixage, génération multiple des
ancrages pour les bibliothèques UML2 et {\ecore}, {\etc}). Ces premiers retours
utilisateur ont joué un rôle important dans l'amélioration de l'outil
{\tomemf}.

Lors de ce test en conditions réelles, l'utilisateur a suivi la méthodologie
avec les stratégies, mais n'a cependant pas utilisé les constructions {\tom}
haut-niveau dédiées aux transformations de modèles. En effet, elles n'étaient
pas incluses dans la version stable du moment.

L'aspect hybride de notre approche a permis à l'utilisateur d'être très
rapidement opérationnel sans qu'il ait à apprendre un nouveau langage complet.
De plus, du fait de son environnement de développement, sa transformation a pu
être immédiatement intégrée dans les outils développés, ce qui n'aurait pas été
aussi simple avec des langages tels que {\atl}, {\kermeta} ou même {\stratego}. Si
notre approche utilisant {\tom} fournit moins de fonctionnalités pour la
transformation de modèles que {\atl}, nous avons l'avantage de ne pas nous
séparer des fonctionnalités (bibliothèques notamment) du langage généraliste
considéré ({\java} dans notre cas), ce qui permet à l'utilisateur d'exprimer
tout de même ce qu'une construction {\tom} ne pourrait pas fournir nativement.

Le principe d'utiliser des stratégies de réécritures pour encoder les
transformations de modèles semble aussi un choix judicieux du point de vue
de l'utilisateur : la décomposition d'une transformation en transformations
élémentaires est naturelle pour appréhender la complexité et le principe
d'écriture de règles est aussi courant dans d'autres langages. Nous ne
changeons donc pas fondamentalement les habitudes du développeur. Grâce aux
mécanismes en jeu dans notre approche, nous avons une forte expressivité dans
les motifs (membre gauche) et nous offrons une grande liberté à l'utilisateur
dans les actions (membre droit) grâce à leur nature composite. 

Le fait que notre approche se fonde sur un changement d'espace technologique
pourrait paraître limitant en première approche (outil supplémentaire pour
opérer le changement), cependant cet outil (générateur d'ancrages formels) est 
extrêmement simple d'utilisation et ne nécessite qu'une seule intervention en
début de développement.

Le choix de l'usage d'un outil par rapport à un autre ne peut être définitif et
universel : il doit être effectué de manière réfléchie en tenant compte de
critère tels que l'environnement de développement, les fonctionnalités
respectives des outils envisagés ainsi que les performances des outils. Dans un
contexte {\java}+{\emf}, compte tenu des premiers retours, le choix de nos
outils est pertinent. 

\section{Performances}

De manière générale, un outil peut être extrêmement intéressant
scientifiquement, mais inutilisable du fait de ses performances (implémentation
inefficace ou problème à résoudre trop complexe). Dans un cadre industriel,
cette question se pose rapidement, étant donné que des modèles \og réels \fg
(généralement de plus grandes tailles que ceux utilisés pour le prototypage
initial) sont transformés. Il est donc naturel d'évaluer ses performances, même
si n'est pas l'objectif premier de l'outil. Dans le domaine de la vérification
du logiciel, les outils de \emph{model-checking} sont généralement des outils
consommateurs de ressources (temps et mémoire), ce qui peut constituer un
goulet d'étranglement dans une chaîne de développement imposant une
vérification de ce type. Il ne faut donc pas que les autres traitements aient
des performances moindres (et deviendraient de fait le nouveau goulet
d'étranglement), ou que leurs temps d'exécution ajoutent trop de délais avant
ce goulet. Un second aspect de l'évaluation des performances vient du fait que
beaucoup d'outils de transformation existent, mais leur passage à l'échelle est
souvent un facteur limitant. C'est donc dans ce contexte que nous avons
effectué une première évaluation des performances de nos outils, d'une part
{\tomemf}, d'autre part les constructions haut-niveau dédiées aux
transformations de modèles.

\subsection{{\tomemf}}

Nous avons expérimenté {\tomemf} sur divers métamodèles afin de nous assurer de
son passage à l'échelle. Très rapidement, nous avons constaté que peu de très
gros métamodèles sont utilisés, les métamodèles {\ecore} et {\uml} étant
souvent les plus gros métamodèles utilisés régulièrement. Nous avons donc
appliqué notre outil sur ces métamodèles pour générer les \emph{mappings}
correspondant, ainsi que sur les parties des deux métamodèles du cas
d'utilisation présenté dans le chapitre~\ref{sec:simplepdl2pn}.

\begin{table}[h]
  \begin{center}
    \begin{tabular}{c|c|cc|c}
  \multirow{2}{*}{Métamodèle} & {\ecore} & \multicolumn{2}{c|}{Mappings {\tom}}
  %& \multirow{2}{*}{Temps moyen \\
  & Temps moyen de \\
  & \#eClassifiers & \#sortes & \#opérateurs & génération (en ms) \\
  \hline
  %Ecore & \num{52} &  \num{56} & \num{28} & \num{248.7} \\
  Ecore & \num{52} &  \num{56} & \num{28} & \num{24} \\
  %UML2 & \num{264} &  \num{345} & \num{346} & \num{1527.6} \\
  UML2 & \num{264} &  \num{345} & \num{346} & \num{1527} \\
  \hline
  %SimplePDL & 16 & x & x & \\
  %DDMM & 4 & 8 & 10 & 256,8\\
%SimplePDL & 4 & 8 & 10 & \num{256.8}\\
SimplePDL & 4 & 8 & 10 & \num{256}\\
  %PetriNet & 14 & x & x & x\\
  %DDMM & 2 & 8 & 8 & 254,5\\
  PetriNet & 2 & 8 & 8 & \num{254}\\
%  PetriNet & 2 & 8 & 8 & \num{254.5}\\
\end{tabular}
%\caption{Tailles de quelques métamodèles significatifs ainsi que des
%  \emph{mappings} {\tomemf} correspondant.}
%\label{table:statsTomEMF}

    \caption{Tailles de quelques métamodèles significatifs ainsi que des
      \emph{mappings} {\tomemf} correspondant.}
    \label{table:statsTomEMF}
  \end{center}
\end{table}

%   nom    : typeterm / op / oplist / oparray / lignes
%SimplePDL : 8 / 8 / 0 / 2 /185 
%PetriNet  : 8 / 6 / 0 / 2 / 185

Nous constatons que les temps de génération donnés dans la
table~\ref{table:statsTomEMF} sont extrêmement bas, de l'ordre de la seconde.
Notre outil de génération d'ancrages est tout à fait capable de gérer un \og
gros \fg{} métamodèle, dans un temps négligeable par rapport à la durée de
développement et d'exécution d'une transformation comme nous allons le voir.
Considérant ces temps de génération pour de tels métamodèles, nous estimons que
les performances de l'outil {\tomemf} sont largement suffisantes et qu'elles ne
sont pas limitantes dans la chaîne de développement.

Outre l'aspect performances que nous jugeons satisfaisant, cet outil s'est
avéré simple à utiliser pour un utilisateur complètement extérieur au projet
{\tom}. Cela nous donne un indice positif sur l'utilisabilité de notre outil.

\subsection{Transformation}
%%\ttodo{chiffres plus à jour ! attention, Benchmark pour SimplePDL, les temps sont exprimés en millisecondes (ms). }

Afin de tester les performances du langage, nous avons utilisé la
transformation \emph{SimplePDLToPetriNet} sur des modèles de tailles
différentes. Nous avons fait ce choix, car ayant servi de support au
développement de nos outils, nous en avions une bonne maîtrise tout en ayant
une confiance élevée dans son implémentation. Nous l'avons aussi implémentée
dans d'autres langages afin d'avoir des points de comparaison avec l'existant.
S'agissant d'une transformation connue dans le domaine, nous avons pu corriger
et améliorer ces implémentations grâce à celles existantes.

Pour le modèle d'entrée, nous avons défini un processus paramétrable généré de
manière déterministe en fonction de deux paramètres :
\begin{itemize}
  \item le nombre de \emph{WorkDefinitions} présentes (\texttt{N}) ;
  \item le fait qu'il y ait ou non des \emph{WorkSequences}.
\end{itemize}

Le processus d'entrée n'est pas hiérarchique, mais cela n'a pas d'impact sur la
complexité du modèle d'entrée étant donné que :
\begin{itemize}
  \item l'image d'un \emph{Process} est plus simple que l'image d'une
    \emph{WorkDefinition}, nous pouvons nous limiter à jouer sur le nombre de
    \emph{WorkDefinitions} plutôt que sur le nombre de \emph{Process} ;
  \item un processus père existant, des éléments \emph{resolve} sont dans tous
    les cas créés pour chaque \emph{WorkDefinition} ;
  \item l'ajout de séquences entre les activités permet d'ajouter d'autres
    éléments \emph{resolve} et donc augmenter la complexité du modèle d'entrée
    et de la transformation. 
\end{itemize}

Il s'agissait surtout d'être en mesure de générer des modèles d'entrée définis
simplement, mais parfaitement maîtrisés (dont les éléments sources sont
dénombrables de manière fiable) et pouvant atteindre de très grosses tailles
(plusieurs millions d'éléments). La figure~\ref{fig:inputModel} donne la forme
générale d'un tel processus donné en entrée ainsi que le réseau de Petri
résultant par l'application de la transformation \emph{SimplePDLToPetriNet}.

\begin{figure}[h]
  \begin{center}
      \begin{tikzpicture}[node distance=1.2cm,>=stealth',bend
  angle=25,auto,scale=1.0,transform shape]

  \tikzstyle{place}=[circle,thick,draw=red!75,fill=red!20,minimum size=5mm]
  \tikzstyle{transition}=[rectangle,thick,draw=blue!75,fill=blue!20,minimum size=4mm]

  \tikzstyle{every label}=[black]

  \begin{scope}
    
    % Petri net A
    \node [place] (pa1) [xshift=-3.5cm]{} ;

    \node [transition] (ta1) [below of=pa1] {}
      edge [pre]  (pa1)
    ;

    %%in order to center tstart transition
    \node [place] (pa) [below of=ta1,circle,draw=white,fill=white] {};

    \node [place] (pa2) [left of=pa] {}
      edge [pre] (ta1)
    ;

    \node [place] (pa3)  [right of=pa] {}
      edge [pre] (ta1)
    ;
    \node at (pa3.west) [left] {};

    \node [transition] (ta2) [below of=pa2] {}
      edge [pre]  (pa2)
    ;

    \node [place] (pa4) [below of=ta2] {}
      edge [pre] (ta2)
    ;

    % Petri net B
    \node [place] (pb1) {};

    \node [transition] (tb1) [below of=pb1] {}
      edge [pre] (pb1)
      ;
    \node at (tb1.east) [right] {};

    %%in order to center tstart transition
    \node [place] (pp) [below of=tb1,circle,draw=white,fill=white] {};
    \node [place] (pb2) [left of=pp] {}
      edge [pre] (tb1)
    ;
    \node [place] (pb3) [right of=pp] {}
      edge [pre] (tb1)
    ;

    \node [transition] (tb2) [below of=pb2] {}
      edge [pre]  (pb2)
      edge [pre,bend right,green!50!black,thick] (pa3)
      ;

    \node [place] (pb4) [below of=tb2] {}
      edge [pre] (tb2)
    ;

    %C, D, ...
    %(finally not invisible) nodes
    \node [place] (pi1) [right of=pb3,draw=white,fill=white] {} ;
    \node [transition] (ti) [below of=pi1,draw=blue!15,fill=blue!5] {}
      edge [pre,bend right,green!50!black,thick] (pb3)
      ;


     % last Petri net: Z
    \node [place] (pz1) [xshift=8cm] {}
    ;

    \node [transition] (tz1) [below of=pz1] {}
      edge [pre] (pz1)
      ;
    \node at (tz1.east) [right] {};

    %%in order to center tstart transition
    \node [place] (pz) [below of=tz1,circle,draw=white,fill=white] {};
    \node [place] (pz2) [left of=pz] {}
      edge [pre] (tz1)
    ;
    %(finally not invisible) nodes
    \node [place] (pi2) [left of=pz2,draw=red!15,fill=red!5] {} ;
    \node (phantom) [left of=pi2,xshift=-0.5cm] {$\cdots\cdots\cdots$};

    \node [place] (pz3) [right of=pz] {}
      edge [pre] (tz1)
    ;

    \node [transition] (tz2) [below of=pz2] {}
      edge [pre,bend right,green!50!black,thick] (pi2)
      edge [pre]  (pz2)
      ;

    \node [place] (pz4) [below of=tz2] {}
      edge [pre] (tz2)
    ;
 
 % Petri net root
    \node [place] (p1) [tokens=1,xshift=-6.5cm]{}
    ;
    
    \node [transition] (t1) [below of=p1] {}
      edge [pre]  (p1)
      edge [post,bend left,dash pattern=on 2pt off 2pt] (pa1)
      edge [post,bend left,dash pattern=on 2pt off 2pt] (pb1)
      edge [post,bend left,dash pattern=on 2pt off 2pt] (pz1)
    ;

    \node [place] (p2) [below of=t1] {}
      edge [pre] (t1)
    ;

    \node [transition] (t2) [below of=p2] {}
      edge [pre]  (p2)
      edge [pre,bend right,dash pattern=on 2pt off 2pt] (pa4)
      edge [pre,bend right,dash pattern=on 2pt off 2pt] (pb4)
      edge [pre,bend right,dash pattern=on 2pt off 2pt] (pz4)
    ;

    \node [place] (p3) [below of=t2] {}
      edge [pre] (t2)
    ;
    
  
  \end{scope}
\end{tikzpicture}

      \caption{Forme générale des modèles d'entrée générés.}
      \label{fig:inputModel}
  \end{center}
\end{figure}

En fonction des deux paramètres et de la définition du processus, nous pouvons
parfaitement dénombrer les éléments constituant le modèle source, ainsi que
ceux constituant le résultat. Il est aussi possible de dénombrer précisément
les éléments intermédiaires \emph{resolve} créés.

Ainsi, dans notre exemple, pour un processus avec $N$ \emph{WorkDefinitions}
($N>1$) chaînées, il y a $N-1$ \emph{WorkSequences}. Les règles de
transformation produisant des réseaux de Petri identiques à ceux vus dans le
chapitre~\ref{sec:simplepdl2pn} que nous rappelons dans la
figure~\ref{fig:rappel2PN}, nous établissons le
tableau~\ref{table:productionRules} donnant les règles de production des
éléments sources. Dans ce tableau, $n$ et $r$ signifient respectivement
$normal$ et $resolve$, les abréviations Pl, Tr, A, P, WD, WS correspondent
respectivement à \emph{Place}, \emph{Transition}, \emph{Arc}, \emph{Process},
\emph{WorkDefinition} et \emph{WorkSequence}. Les trois dernières colonnes du
tableau sont les totaux d'éléments \emph{resolve} créés (T$_{resolve}$),
d'éléments créés (T$_{\text{\textit{intermédiaire}}}$), et d'éléments dans les
versions finales des images des sources après résolution (T$_{final}$).

\begin{figure}[!h]
  \begin{center}
    \begingroup
\tikzset{every picture/.style={scale=0.7}}%
    \begin{subfigure}[h]{0.25\linewidth}
          \begin{tikzpicture}[node distance=1.1cm,>=stealth',bend
  angle=45,auto,scale=1.0,transform shape]

  \tikzstyle{place}=[circle,thick,draw=red!75,fill=red!20,minimum size=5mm]
  \tikzstyle{transition}=[rectangle,thick,draw=blue!75,
  			  fill=blue!20,minimum size=4mm]

  \tikzstyle{every label}=[black]

  \begin{scope}
    % Petri net  Process
    \node [place] (ppc1) [tokens=1] [xshift=0cm]{}
    ;
    \node at (ppc1.west) [left] {};
    
    \node [transition] (tp1) [right of=ppc1,dash pattern=on 2pt off 2pt]  {}
      edge [post,bend right,dash pattern=on 2pt off 2pt] (ppc1)
    ;
    \node at (tp1.south) [below] {};

    \node [transition] (tpc1) [below of=ppc1] {}
      edge [pre]  (ppc1)
    ;
    \node at (tpc1.west) [left] {};

    \node [place] (ppc2) [below of=tpc1] {}
      edge [pre] (tpc1)
    ;
    \node at (ppc2.west) [left] {};

    \node [transition] (tpc2) [below of=ppc2] {}
      edge [pre]  (ppc2)
    ;
    \node at (tpc2.west) [left] {};

    \node [place] (ppc3) [below of=tpc2] {}
      edge [pre] (tpc2)
    ;
    \node at (ppc3.west) [left] {};
    
    \node [transition] (tp2) [right of=ppc3,dash pattern=on 2pt off 2pt] {}
      edge [pre,bend left,dash pattern=on 2pt off 2pt] (ppc3)
    ;
    \node at (tp2.north) [above] {};
  \end{scope}

\end{tikzpicture} 

      %\caption{P P P}
      \label{fig:rappelP2PN}
    \end{subfigure}
    \quad %\qquad
    \begin{subfigure}[h]{0.30\linewidth}
      \begin{tikzpicture}[node distance=1.3cm,>=stealth',bend
  angle=45,auto,scale=1.0,transform shape]

  \tikzstyle{place}=[circle,thick,draw=red!75,fill=red!20,minimum size=5mm]
  \tikzstyle{transition}=[rectangle,thick,draw=blue!75,
  			  fill=blue!20,minimum size=4mm]

  \tikzstyle{every label}=[black]

  \begin{scope}
    % Petri net A
    \node [place] (p1) [tokens=1] [xshift=-5cm]{}
    ;
    \node at (p1.west) [left] {};

    \node [transition] (tp1) [right of=p1,dash pattern=on 2pt off 2pt]  {}
      edge [post,bend right,dash pattern=on 2pt off 2pt] (p1)
    ;
    \node at (tp1.south) [below] {};

    \node [transition] (t1) [below of=p1] {}
      edge [pre]  (p1)
    ;
    \node at (t1.west) [left] {};

    %%in order to center tstart transition
    \node [place] (p) [below of=t1,circle,draw=white,fill=white] {};

    \node [place] (p2) [left of=p] {}
      edge [pre] (t1)
    ;
    \node at (p2.west) [left] {};
    \node [place] (p3)  [right of=p] {}
      edge [pre] (t1)
    ;
    \node at (p3.west) [left] {};

    \node [transition] (t2) [below of=p2] {}
      edge [pre]  (p2)
    ;
    \node at (t2.west) [left] {};

    \node [place] (p4) [below of=t2] {}
      edge [pre] (t2)
    ;
    \node at (p4.west) [left] {};
    
    \node [transition] (tp2) [right of=p4,dash pattern=on 2pt off 2pt] [xshift=0.5cm] {}
      edge [pre,bend left,dash pattern=on 2pt off 2pt] (p4)
    ;
    \node at (tp2.north) [above] {};

  \end{scope}

\end{tikzpicture}

      %\caption{WD WD WD}
      \label{fig:rappelWD2PN}
    \end{subfigure}
    \quad %\qquad
    \begin{subfigure}[h]{0.35\linewidth}
      \begin{tikzpicture}[node distance=1.2cm,>=stealth',bend
  angle=25,auto,scale=1.0,transform shape]

  \tikzstyle{place}=[circle,thick,draw=red!15,fill=red!5,minimum size=5mm]
  \tikzstyle{transition}=[rectangle,thick,draw=blue!15,fill=blue!5,minimum size=4mm]

  \tikzstyle{edge}=[black!25!black!25]
  \tikzstyle{every label}=[black]

  \begin{scope}
    % Petri net A
    \node [place] (p1) [tokens=0] [xshift=-3.5cm]{}
    ;

    \node [transition] (t1) [below of=p1] {}
      edge [pre,black!25]  (p1)
    ;

    %%in order to center tstart transition
    \node [place] (p) [below of=t1,circle,draw=white,fill=white] {};

    \node [place] (p2) [left of=p] {}
      edge [pre,black!25] (t1)
    ;

    \node [place] (p3)  [right of=p,draw=red!75,fill=red!20,dash pattern=on 2pt off 2pt] {}
      edge [pre,black!25] (t1)
    ;
    \node at (p3.west) [left] {};

    \node [transition] (t2) [below of=p2] {}
      edge [pre,black!25]  (p2)
    ;

    \node [place] (p4) [below of=t2] {}
      edge [pre,black!25] (t2)
    ;

    % Petri net B
    \node [place] (pb1) [tokens=0] {}
    ;

    \node [transition] (tb1) [below of=pb1,draw=blue!75,fill=blue!20,dash pattern=on 2pt off 2pt] {}
      edge [pre,black!25] (pb1)
      edge [pre,bend right,green!50!black,thick] (p3)
      ;
    \node at (tb1.east) [right] {};

    %%in order to center tstart transition
    \node [place] (pp) [below of=tb1,circle,draw=white,fill=white] {};
    \node [place] (pb2) [left of=pp] {}
      edge [pre,black!25] (tb1)
    ;
    \node [place] (pb3) [right of=pp] {}
      edge [pre,black!25] (tb1)
    ;

    \node [transition] (tb2) [below of=pb2] {}
      edge [pre,black!25]  (pb2)
      ;

    \node [place] (pb4) [below of=tb2] {}
      edge [pre,black!25] (tb2)
    ;
  \end{scope}
\end{tikzpicture}

      %\caption{WS WS WS}
      \label{fig:rappelWD2PN}
    \end{subfigure}
    \caption{Réseaux de Petri images d'un \emph{Process}, d'une
    \emph{WorkDefinition} et d'une \emph{WorkSequence}.}
    \label{fig:rappel2PN}
\endgroup
  \end{center}
\end{figure}

Compte tenu du fait que nos modèles d'entrée générés n'ont pas de processus
hiérarchiques, les arcs de synchronisation ainsi que les éléments
\emph{resolve} associés n'existent pas dans l'image du \emph{Process}.

\begin{table}[h]
  \begin{center}
    \begin{tabular}[h]{c|c|c|c|c|c|c|c|c}
  \multirow{2}{*}{Source} & \multicolumn{2}{c|}{Places} &
  \multicolumn{2}{c|}{Transitions} & \multirow{2}{*}{Arcs} &
  \multirow{2}{*}{T$_{resolve}$} &
  \multirow{2}{*}{T$_{\text{\textit{intermédiaire}}}$} &
  \multirow{2}{*}{T$_{final}$} \\
  & $n$ & $r$ & $n$ & $r$ & & & & \\
  \hline
  P  & 3 & 0 & 2 & 0 & 4 & 0 & 9 & 9 \\
  WD & 4 & 0 & 2 & 2 & 7 & 2 & 15 & 13 \\
  WS & 0 & 1 & 0 & 1 & 1 & 2 & 3 & 1 \\
\end{tabular}
%\caption{Dénombrement des éléments cibles créés à partir d'éléments sources}
%\label{table:productionRules}

    \caption{Dénombrement des éléments cibles créés à partir d'éléments
    sources.}
    \label{table:productionRules}
  \end{center}
\end{table}


Ainsi, pour un processus composé de $N$ \emph{WorkDefinitions} (avec $N > 1$)
on obtient $14N+8$ éléments cibles \emph{normaux} et $4N-2$ éléments
\emph{resolve}, comme le résume la table~\ref{table:denombrementCibles}.

\begin{table}[h]
  \begin{center}
    %\begin{itemize}
%  \item $4N+3$ \emph{Places} normales et $N-1$ \emph{resolve} ;
%  \item $2N+2$ \emph{Transitions} normales et $3N-1$ \emph{resolve} ;
%  \item $8N+3$ \emph{Arcs}.
%%    exemple \#1 : 1 processus composé de N WorkDefinitions, toutes chaînées
%%    en f2s ; 2N éléments src (1 P + N WD + (N-1) WS ; 14N+8 éléments tgt (P :
%%    3pl+2tr+4a ; WD : 4pl+2tr+5a ; WS : 1a ; arcs de synchro : 2/WD) ; 
%%    4N-2 éléments resolve (WD : 2tr ; WS : 1pl+1tr)
%%%  \item exemple \#2 : ???
%\end{itemize}
\begin{tabular}[h]{c|c|c|c|c|c|c}
  \multicolumn{2}{c|}{Places} &
  \multicolumn{2}{c|}{Transitions} & \multirow{2}{*}{Arcs} &
  \multirow{2}{*}{T$_{resolve}$} &
  \multirow{2}{*}{T$_{normal}$} \\
  $n$ & $r$ & $n$ & $r$ & & & \\
  \hline
  $4N+3$ & $N-1$ & $2N+2$ & $3N-1$ & $8N+3$ & $4N-2$ & $14N+8$ \\
\end{tabular}
%\caption{Dénombrement des éléments cibles créés en fonction du nombre de
%  \emph{WorkDefinitions} donné en entrée ($N$).}
%\label{table:denombrementCibles}

    \caption{Dénombrement des éléments cibles créés en fonction du nombre de
      \emph{WorkDefinitions} donné en entrée ($N>1$).}
    \label{table:denombrementCibles}
  \end{center}
\end{table}

Les valeurs que nous donnons par la suite sont celles d'expériences menées sur
un serveur ayant les caractéristiques suivantes :
\begin{itemize}
  \item système Unix 64 bits ;
  \item RAM : 24 GB ;
  \item processeurs : 2 $\times$ \num{2.93} GHz Quad-Core Intel Xeon.
\end{itemize}
Cependant, mis à part pour les transformations à plusieurs millions
d'éléments qui demandent des ressources supérieures à celles fournies par un
poste de bureau, ces expériences peuvent tout à fait être reproduites sur une
machine bien plus modeste.

La table~\ref{table:tempsTomFull} donne les temps moyens de la transformation
appliquée sur des modèles de tailles différentes. La première colonne donne le
nombre d'éléments sources du modèle d'entrée en fonction du paramètre d'entrée
$N$. La deuxième colonne donne le nombre d'éléments constituant le modèle
résultant. Les temps moyens sont donnés par les colonnes suivantes, en séparant
les deux phases, la dernière colonne donnant les durées totales.  Pour donner
une idée de la taille des données, la sérialisation \texttt{.xmi} d'un modèle à
\num{2000000} d'éléments sources (dernière ligne) est un fichier d'environ
\num{320} Mo.

%La table~\ref{table:denombrement} donne les tailles des modèles sources et
%cibles, ainsi que les nombres d'éléments de notre exemple. Le modèle source
%était entièrement généré, et connaissant sa forme, nous sommes en mesure de
%dénombrer ses éléments ainsi que ceux du modèle résultant. Pour donner une idée
%de la taille des données, la sérialisation \texttt{.xmi} d'un modèle à 2000000
%d'éléments est un fichier d'environ 320 Mo.

%\begin{table}[h]
%\begin{center}
%\begin{tabular}{rrrr||rr|rr|r|r|r}
  P & WD & WS & Sources & \multicolumn{2}{c|}{pl} & \multicolumn{2}{c|}{tr} &
  arcs & total inter. & total final \\
    & & & & n & res & n & res & & & \\
  \hline
  1 & 1    & 0   & 2    & 7    & 0   & 4    & 2    & 11   & 24    & 22 \\
  1 & 2    & 1   & 4    & 11   & 1   & 22   & 29   & 19   & 42    & 36 \\
  1 & 10   & 9   & 20   & 43   & 9   & 22   & 29   & 83   & 186   & 148 \\
  1 & 100  & 99  & 200  & 403  & 99  & 202  & 299  & 803  & 1806  & 1408 \\
  1 & 500  & 499 & 1000 & 2003 & 499 & 1002 & 1499 & 4003 & 9006  & 7008 \\
  1 & 1000 & 999 & 2000 & 4003 & 999 & 2002 & 2999 & 8003 & 18006 & 14008 \\
  1 & 2000 & 1999 & 4000 & 8003 & 1999 & 4002 & 2999 & 16003 & 36006 & 28008 \\
  1 & 3000 & 2999 & 6000 & 12003 & 2999 & 6002 & 2999 & 24003 & 54006 & 42008 \\
  1 & 4000 & 3999 & 8000 & 16003 & 3999 & 8002 & 2999 & 32003 & 72006 & 56008 \\
  1 & 5000 & 4999 & 10000& 20003& 4999& 10002& 14999& 40003& 90006 & 70008 \\
  1 & 10000& 9999 & 20000& 40003& 9999& 20002& 29999& 80003& 180006& 140008 \\
  1 & 20000& 19999& 40000& 80003& 19999& 400002& 29999& 160003& 360006& 280008 \\
  1 & 50000& 49999 & 100000& 200003& 49999& 100002& 29999& 400003& 900006& 700008 \\
  1 & 100000& 99999& 200000& 400003& 99999& 200002& 29999& 800003& 1800006&
  1400008 \\
  1 & 500000&  & & & & & & & &  \\
  1 & 1000000& 999999& 2000000& 4000003& 999999& 2000002& 29999& 8000003&
  18000006& 14000008 \\
\end{tabular}
%\caption{Dénombrement des éléments IN et OUT pour l'exemple 1}
%\label{table:denombrement}

%\caption{Dénombrement des éléments IN et OUT pour l'exemple 1}
%\label{table:denombrement}
%\end{center}
%\end{table}


%OSEF, on fait du motu-one
%Version \og filip \fg :
%\begin{table}[h]
%\begin{center}
%\begin{tabular}{rr|r|rrrr|r}
%  N (nbr WD) & src(2N) & tgt(14N+8)&
%  \multicolumn{2}{c}{phase\#1} & \multicolumn{2}{c}{phase\#2}  & total \\
%          & & & temps & \% & temps & \% & \\
%%  Exemple & N (nbr WD) & tot. src(2N) & tot. tgt(14N+8)& phase\#1 &  & total \\
%  \hline
%  10 & 20 & 148 & 19 & 19.19 & 80 & 80.81 & 99 \\
%  100 & 200 & 1408 & 76 & 7.79 & 891 & 92.21 & 976 \\
%  500 & 1000 & 7008 & 249 & 0.54 & 45768 & 99.46 & 46017 \\
%  1000 & 2000 & 14008 & 522 & 0.12 & 437047 & 99.88 & 437569 \\
%       &      &       &     & & \multicolumn{2}{r|}{$\sim$7min 17s} & $\sim$7min 17s \\
%  5000 & 10000 & 70008 & ?? & 0.?? & ?? & 99.?? & ?? \\
%       &      &       &     & & \multicolumn{2}{r|}{$\sim$??min ??s} & $\sim$??min ??s \\
%  10000 & 20000 & 140008 & ?? & 0.01 & ?? & 99.99 & ?? \\
%        &       &        &    & & \multicolumn{2}{r|}{$\sim$??min ??s} & $\sim$??min ??s \\
%\end{tabular}
%\caption{Temps de transformation, exemple 1}
%\label{table:stats}
%\end{center}
%\end{table}

\begin{table}[h]
  \begin{center}
    %La table~\ref{table:tempsTomFull} donne les temps moyens 
%%%%%%%%%%%%%%%%%%%%
%old
%\begin{table}[h]
%\begin{center}
%\begin{tabular}{rr|r|rrrr|r}
%  N (nbr WD) & src(2N) & tgt(14N+8)&
%  \multicolumn{2}{c}{phase\#1} & \multicolumn{2}{c}{phase\#2}  & total \\
%          & & & temps & \% & temps & \% & \\
%%  Exemple & N (nbr WD) & tot. src(2N) & tot. tgt(14N+8)& phase\#1 &  & total \\
%  \hline
%  10 & 20 & 148 & 15 & 18.52 & 66 & 81.48 & 81 \\
%  100 & 200 & 1408 & 65 & 9.60 & 612 & 90.40 & 677 \\
%  500 & 1000 & 7008 & 198 & 0.53 & 37470 & 99.47 & 37668 \\
%  1000 & 2000 & 14008 & 343& 0.12 & 281482 & 99.88 & 281825 \\
%       &      &       &     & & \multicolumn{2}{r|}{$\sim$4min 41s} & $\sim$4min 41s \\
%  5000 & 10000 & 70008 & ?? & 0.?? & ?? & 99.?? & ?? \\
%       &      &       &     & & \multicolumn{2}{r|}{$\sim$??min ??s} & $\sim$??min ??s \\
%  10000 & 20000 & 140008 & ?? & 0.01 & ?? & 99.99 & ?? \\
%        &       &        &    & & \multicolumn{2}{r|}{$\sim$??min ??s} & $\sim$??min ??s \\
%\end{tabular}
%\caption{Temps de transformation, exemple 1}
%\label{table:tempsTomFull}
%\end{center}
%\end{table}
%
%%%%%%%%%%%%%%%%%%%%
%%V1
%\begin{tabular}{rr|r|rrrr|r}
%%  N (nbr WD) & src(2N) & tgt(14N+8)&
%  N (\#WD) & \#src & \#tgt&
%  \multicolumn{2}{c}{phase 1} & \multicolumn{2}{c|}{phase 2} & \multicolumn{1}{c}{Total} \\
%          & & & temps & \% & temps & \% & \\
%%  Exemple & N (nbr WD) & tot. src(2N) & tot. tgt(14N+8)& phase\#1 &  & total \\
%  \hline
%  2       & 4       & 36       & \todo{0}         & \todo{\#DIV/0!} & \todo{0}        &
%  \todo{\#DIV/0!} & \todo{0}        \\
%\num{10}      & \num{20}      & \num{148}      & \num{16.7}      & \num{97.09}   & \num{0.5}      & \num{2.91}    & \num{17.2}     \\
%\num{100}     & \num{200}     & \num{1408}     & \num{66}        & \num{90.66}   & \num{6.8}      & \num{9.34}    & \num{72.8}     \\
%\num{500}     & \num{1000}    & \num{7008}     & \num{208.6}     & \num{87.91}   & \num{28.7}     & \num{12.09}   & \num{237.3}    \\
%\num{1000}    & \num{2000}    & \num{14008}    & \num{359.6}     & \num{86.94}   & \num{54}       & \num{13.06}   & \num{413.6}    \\
%\num{2000}    & \num{4000}    & \num{28008}    & \num{833.4}     & \num{85.38}   & \num{142.7}    & \num{14.62}   & \num{976.1}    \\
%%        & & & \multicolumn{2}{|r}{$\sim$??min ??s} & \multicolumn{2}{r|}{$\sim$??min ??s} & $\sim$0min 1s \\
%\num{3000}    & \num{6000}    & \num{42008}    & \num{1470}      & \num{86.79}   & \num{223.8}    & \num{13.21}   & \num{1693.8}   \\
%%        & & & \multicolumn{2}{|r}{$\sim$??min ??s} & \multicolumn{2}{r|}{$\sim$??min ??s} & $\sim$0min 2s \\
%\num{4000}    & \num{8000}    & \num{56008}    & \num{2409.4}    & \num{87.56}   & \num{342.4}    & \num{12.44}   & \num{2751.8}   \\
%%        & & & \multicolumn{2}{|r}{$\sim$??min ??s} & \multicolumn{2}{r|}{$\sim$??min ??s} & $\sim$0min 3s \\
%\num{5000}    & \num{10000}   & \num{70008}    & \num{3610.2}    & \num{88.02}   & \num{491.4}    & \num{11.98}   & \num{4101.6}   \\
%%        & & & \multicolumn{2}{|r}{$\sim$??min ??s} & \multicolumn{2}{r|}{$\sim$??min ??s} & $\sim$0min 4s \\
%\num{10000}   & \num{20000}   & \num{140008}   & \num{13436.8}   & \num{88.42}   & \num{1760}     & \num{11.58}   &
%\num{15196.8}  \\
%        & & & \multicolumn{2}{r}{$\sim$0min 13s} & \multicolumn{2}{r|}{$\sim$0min 2s} & $\sim$0min 15s \\
%        \num{20000}   & \num{40000}   & \num{280008}   & \num{52300.5}   & \num{88.52}   & \num{6781.2}   & \num{11.48}   &
%\num{59081.7}  \\
%        & & & \multicolumn{2}{r}{$\sim$0min 52s} & \multicolumn{2}{r|}{$\sim$0min 7s} & $\sim$0min 59s \\
%      \num{50000}   & \num{100000}  & \num{700008}   & \num{322164.1}  & \num{88.54}   & \num{41708.1}  & \num{11.46}   &
%\num{363872.2} \\
%        & & & \multicolumn{2}{r}{$\sim$5min 22s} & \multicolumn{2}{r|}{$\sim$0min 42s} & $\sim$6min 4s \\
%      \num{100000}  & \num{200000}  & \num{1400008}  & \num{1250304.9} & \num{88.22}   & \num{166934.2} & \num{11.78}   &
%\num{1417239.1}\\
%        & & & \multicolumn{2}{r}{$\sim$20min 50s} & \multicolumn{2}{r|}{$\sim$2min 47s} & $\sim$23min 37s \\
%        \num{1000000} & \num{2000000} & \num{14000008} & \todo{\num{126094099}}
%        %& \todo{\num{83.86}} & \todo{\num{24262874}}        &
%        & \todo{\num{86.91}} & \todo{\num{154815717}}        &
%        %\todo{\num{16.14}} & \todo{\num{150356973}}      \\
%        \todo{\num{13.09}} & \todo{\num{23323160}}      \\
%        %& & & \multicolumn{2}{|r}{$\sim$35h} & \multicolumn{2}{r|}{$\sim$6h 45min} & $\sim$41h 45min \\
%        & & & \multicolumn{2}{r}{$\sim$43h} & \multicolumn{2}{r|}{$\sim$6h 30min} & $\sim$49h 30min \\
%\end{tabular}
%%%%%%%%%%%%%%%%%%%%
%%V2
\begin{tabular}{r|r|rrrr|r}
%  N (nbr WD) & src(2N) & tgt(14N+8)&
   \#src (2N) & \#tgt& \multicolumn{2}{c}{phase 1} & \multicolumn{2}{c|}{phase 2} & \multicolumn{1}{c}{Total} \\
          & & temps & \% & temps & \% & \\
%  Exemple & N (nbr WD) & tot. src(2N) & tot. tgt(14N+8)& phase\#1 &  & total \\
  \hline
% 4       & 36       & \todo{0}         & \todo{\#DIV/0!} & \todo{0}        &  \todo{\#DIV/0!} & \todo{0}        \\
 %\num{20}      & \num{148}      & \num{16.7}      & \num{97.09}   & \num{0.5}      & \num{2.91}    & \num{17.2}     \\
 \num{20} & \num{148} & \num{16.7}ms & \num{97.09} & \num{0.5}ms & \num{2.91} &
 \num{17.2}ms \\
 \num{200} & \num{1408} & \num{66}ms & \num{90.66} & \num{6.8}ms & \num{9.34} &
 \num{72.8}ms \\
 \num{1000} & \num{7008} & \num{208}ms & \num{87.76} & \num{29}ms & \num{12.24} & \num{237}ms \\
 \num{2000} & \num{14008} & \num{359}ms & \num{86.92} & \num{54}ms & \num{13.08} & \num{413}ms \\
 \num{4000} & \num{28008} & \num{833}ms & \num{85.35} & \num{143}ms & \num{14.65} & \num{976}ms\\
 \num{6000} & \num{42008} & $\sim$\num{1.47}s & \num{86.98} & \num{223}ms & \num{13.02}   & $\sim$\num{1.69}s \\
 \num{8000} & \num{56008} & $\sim$\num{2.4}s & \num{87.27} & \num{342}ms & \num{12.73} & $\sim$\num{2.75}s \\
 \num{10000} & \num{70008} & $\sim$\num{3.61}s & \num{88.03} & \num{491}ms & \num{11.97}   & \num{4.1}s \\
 \num{20000} & \num{140008} & $\sim$\num{13.4}s & \num{88.16} & $\sim$\num{1.8}s & \num{11.84} & $\sim$\num{15.2}s \\
 \num{40000} & \num{280008} & $\sim$0min 52s & \num{88.14} & $\sim$0min 7s & \num{11.86}   & $\sim$0min 59s \\
 \num{100000} & \num{700008} & $\sim$5min 22s & \num{88.54} & $\sim$0min 42s & \num{11.46}   & $\sim$6min 4s \\
 \num{200000} & \num{1400008} & $\sim$20min 50s & \num{88.22} & $\sim$2min 47s & \num{11.78}   & $\sim$23min 37s\\
 \num{2000000} & \num{14000008} & $\sim$35h 40min
        %& \todo{\num{83.86}} & \todo{\num{24262874}}        &
        & \num{84.87} & $\sim$6h 30min       &
        %\todo{\num{16.14}} & \todo{\num{150356973}}      \\
        \num{11.78} &  $\sim$42h 10min     \\
        %& & & \multicolumn{2}{|r}{$\sim$35h} & \multicolumn{2}{r|}{$\sim$6h 45min} & $\sim$41h 45min \\
\end{tabular}
%%%%%%%%%%%%%%%%%%%%
%\caption{Mesures de durées de transformation en fonction des tailles des
%modèles sources.}
%\label{table:tempsTomFull}

    \caption{Mesures de durées de transformation en fonction des tailles des
    modèles sources.}
%    \label{table:temps}
    \label{table:tempsTomFull}
  \end{center}
\end{table}

% 1er décembre : 126 094 099 + 24 262 874 = 150 356 973 (model gen: 57 877 925)
%               35h1min34s + 6h44min..s = 41h45min (16h4min37s)
%benchWOemfresolve/MY

%? 53644826 -> 14h 54min

La figure~\ref{fig:courbeTemps} donne une représentation graphique des temps
d'exécution en fonction de la taille du modèle d'entrée. Plus le modèle
source est gros, plus la durée de la transformation est élevée. Nous notons que
plus le modèle source est important, plus la part de la première phase de la
transformation est importante par rapport à la phase de résolution. Cela
s'explique par deux facteurs : d'une part par la transformation elle-même qui
crée bien plus d'éléments cibles (\emph{normaux} et \emph{resolve}) qu'elle ne
doit en résoudre, d'autre part par les améliorations et optimisations que nous
avons apportées au code généré et aux mécanismes développés.

\begin{figure}[h]
  \begin{center}
    \definecolor{mygreen}{HTML}{129d1c}
\pgfplotsset{scaled y ticks=false,scaled x ticks=false}
\begin{tikzpicture}
  \begin{axis}[
      xlabel=$Nombre\ de\ sources$,%0 -> 250 000
      ylabel=$Temps\ moyen\ (en\ ms)$, %0 -> 1 600 000
      %ylabel style={xshift=-5pt,yshift=15pt},
      ylabel style={yshift=20pt},
      legend cell align=left,
      legend style={draw=none,legend pos=north west,font=\small},
      xticklabel style={/pgf/number format/fixed},
      yticklabel style={/pgf/number format/fixed},
%      ytick={aaa,1000000,1500000},
     /pgf/number format/1000 sep={\ }]

     \addplot[color=mygreen,mark=+] plot coordinates {
%      (4,      )
      (20,     16.7)
      (200,    66)
      (1000,   208.6)
      (2000,   359.6)
      (4000,   833.4)
      (6000,   1470)
      (8000,   2409.4)
      (10000,  3610.2)
      (20000,  13436.8)
      (40000,  52300.5)
      (100000, 322164.1)
      (200000, 1250304.9)
%      (2000000, 131216793.1)
    } ;
    %} node[pos=0.8,pin={$P_1$},xshift=0.5cm,yshift=-0.5cm] {} ;
    %} node[pos=0.1,pin=135:{\color{purple}$A$}] {} ;
    %node[pos=0.54](endofplotsquare){} ;
    %\node [right, color=mygreen] at (endofplotsquare) {A};
    %\node at (200000,300000) {\color{mygreen}$P_1$};
    \node at (210,125) {\color{mygreen}$P_1$};
    \addlegendentry{({\color{mygreen}$P_1$}) Temps moyen phase \#1}

    \addplot[color=blue,mark=+] plot coordinates {
%      (4,      )
      (20,     0.5)
      (200,    6.8)
      (1000,   28.7)
      (2000,   54)
      (4000,   142.7)
      (6000,   223.8)
      (8000,   342.4)
      (10000,  491.4)
      (20000,  1760)
      (40000,  6781.2)
      (100000, 41708.1)
      (200000, 166934.2)
%      (2000000, 2338520.8)
    } ;
    %} node[pos=1,pin={$P_2$}] {} ;
    \node at (210,17) {\color{blue}$P_2$};
    %\node[pin=Toto $P_2$] at (axis cs:200000,300000) {};
    \addlegendentry{({\color{blue}$P_2$}) Temps moyen phase \#2}

    \addplot[color=red,mark=+] plot coordinates {
%      (4,      )
      (20,     17.2)
      (200,    72.8)
      (1000,   237.3)
      (2000,   413.6)
      (4000,   976.1)
      (6000,   1693.8)
      (8000,   2751.8)
      (10000,  4101.6)
      (20000,  15196.8)
      (40000,  59081.7)
      (100000, 363872.2)
      (200000, 1417239.1)
 %     (2000000, 154602033.9)
  } ;
  %} node[pos=0.9,pin={$T$}] {} ;
    \node at (210,142) {\color{myred}$T$};
    \addlegendentry{({\color{myred}$T$})~ Temps moyen total}

    %\legend{$d=2$\\$d=3$\\$d=4$\\$d=5$\\$d=6$\\}
  \end{axis}
\end{tikzpicture}

  \caption{Temps moyen de transformation (en ms) en fonction du nombre d'éléments
  dans le modèle source (phase 1, phase 2, total).}
  \label{fig:courbeTemps}
  \end{center}
\end{figure}

En effet, ces résultats ne sont pas les résultats que nous pourrions obtenir
avec la version stable actuelle de nos outils : initialement, la phase de
résolution était largement plus consommatrice de ressources et la durée d'une
transformation était quasiment celle de la seconde phase (environ $99\%$ du
temps d'exécution pour des modèles de taille $>$100). De plus, dans leur
première version stable publiée, outre une certaine lenteur, nos outils
consomment une grande quantité de mémoire. Lors des premières expériences, nos
transformations étaient beaucoup plus longues à s'exécuter que la même
transformation développée en {\atl} (donnée en~\ref{annexe:atlpdl2pn}). Nous ne
pouvions exécuter une transformation d'un modèle source de \num{20000} éléments
ou plus, par manque de mémoire (20GB alloués par défaut). De son côté, {\atl}
franchissait ce seuil est bloquait aux alentours de \num{50000} éléments
sources par manque de mémoire. C'est ce qu'illustre le
tableau~\ref{table:tempsTomEvolution} qui donne les temps d'exécution de deux
versions de nos outils, que nous comparons à ceux avec {\atl}. Nous avons donc
travaillé en priorité sur l'amélioration du code généré pour la phase de
résolution afin de corriger ces lacunes. 

\begin{table}[h]
  \begin{center}
    %FIXME
\begin{tabular}{r|r|r|rr|rr}
%FIXME N & \#src (2N) & \#tgt & \multicolumn{2}{c|}{Tom} & \multicolumn{2}{c}{ATL} \\
N & \#src (2N) & \#tgt & \multicolumn{2}{c|}{Tom} & ATL \\
%FIXME  &            &       & v1 & v2                  & Simple  & Parallèle \\
  &            &       & v1 & v2                  & Simple \\
\hline
\num{10}     & \num{20}     & \num{148}     & \num{83}ms & \num{17}ms     & - \\%& - \\
\num{100}    & \num{200}    & \num{1408}    & \num{638}ms & \num{73}ms    & - \\%& - \\
\num{500}    & \num{1000}   & \num{7008}    & $\sim$38s & \num{237}ms    & - \\%& - \\
\num{1000}   & \num{2000}   & \num{14008}   & $\sim$4min 14s & \num{414}ms & - \\%& - \\
\num{2000}   & \num{4000}   & \num{28008}   & $\sim$36min 55s & \num{976}ms & - \\%& - \\
\num{3000}   & \num{6000}   & \num{42008}   & $\sim$1h 59min & \num{1.7}s & - \\%& - \\
\num{4000}   & \num{8000}   & \num{56008}   & $\sim$4h 24min & \num{2.8}s & - \\%& - \\
\num{5000}   & \num{10000}  & \num{70008}   & $\sim$8h 14min & \num{4.1}s & $\sim$28s\\%& - \\
\num{10000}  & \num{20000}  & \num{140008}  & N/A & $\sim$0min 15s & $\sim$1min 41s\\%& - \\
\num{20000}  & \num{40000}  & \num{280008}  & N/A & $\sim$0min 59s & $\sim$7min 38s\\%& - \\
\num{50000}  & \num{100000} & \num{700008}  & N/A & $\sim$6min 4s  & $\sim$1h 39min\\%& - \\
% & & & & & \todo{$\sim$1h 2min 11s}& \\
% & & & & & \todo{$\sim$1h 53s}& \\

\num{100000} & \num{200000} & \num{1400008} & N/A & $\sim$23min 37s & $\sim$\\%& \\
%FIXME\num{1000000}& \num{2000000}& \num{14000008}& N/A & \todo{$\sim$42h 57min} & $\sim$& \\
\num{1000000}& \num{2000000}& \num{14000008}& N/A & $\sim$42h 10min & $\sim$\\%& \\
\end{tabular}
%\caption{Comparaison des performances entre Tom (première et dernière versions) et ATL}
%\label{table:tempsTomEvolution}
%
%
%
%\begin{tabular}{r|r|r|rr|rr}
%N & \#src (2N) & \#tgt & \multicolumn{2}{c|}{Tom} & \multicolumn{2}{c}{ATL} \\
%  &            &       & v1 & v2                  & filip  & motu \\
%\hline
%\num{10}     & \num{20}     & \num{148}     & \num{83}ms & \num{17}ms     & $<$1s & \\
%\num{100}    & \num{200}    & \num{1408}    & \num{638}ms & \num{73}ms    & $<$1s & \\
%\num{500}    & \num{1000}   & \num{7008}    & $\sim$38s & \num{237}ms    & $\sim$5s & \\
%\num{1000}   & \num{2000}   & \num{14008}   & $\sim$4min 14s & \num{414}ms & $\sim$7s & \\
%\num{2000}   & \num{4000}   & \num{28008}   & $\sim$36min 55s & \num{976}ms & $\sim$21s & \\
%\num{3000}   & \num{6000}   & \num{42008}   & $\sim$1h 59min & \num{1.7}s & $\sim$41s & \\
%\num{4000}   & \num{8000}   & \num{56008}   & $\sim$4h 24min & \num{2.8}s & $\sim$1min 4s & \\
%\num{5000}   & \num{10000}  & \num{70008}   & $\sim$8h 14min & \num{4.1}s & $\sim$1min 56s & $\sim$28s\\
%\num{10000}  & \num{20000}  & \num{140008}  & N/A & $\sim$0min 15s   & $\sim$9min 14s & $\sim$1min 41s\\
%\num{20000}  & \num{40000}  & \num{280008}  & N/A & $\sim$0min 59s   & $\sim$35min 14s & $\sim$7min 38s \\
%\num{50000}  & \num{100000} & \num{700008}  & N/A & $\sim$6min 4s  & N/A  & \todo{$\sim$1h 39min 18s}\\
% & & & & & & \todo{$\sim$1h 2min 11s}\\
% & & & & & & \todo{$\sim$1h 53s}\\
%\num{100000} & \num{200000} & \num{1400008} & N/A & $\sim$23min 37s & N/A  & $\sim$\\
%\num{1000000}& \num{2000000}& \num{14000008}& N/A & \todo{$\sim$49h 30min} & N/A & $\sim$\\
%\end{tabular}

    \caption{Comparaison des performances entre Tom (première et dernière
    versions) et ATL.}
    \label{table:tempsTomEvolution}
  \end{center}
\end{table}

Dans un premier temps, constatant que l'essentiel du temps était passé dans
l'appel de fonctions du \emph{framework} {\emf}, nous avons modifié la
résolution de liens. Nous nous reposions exclusivement sur les méthodes
\emph{emf} gérant la résolution de liens, qui étaient appelées au sein de la
stratégie de résolution. Cependant, l'implémentation de cette méthode dans le
\emph{framework} {\emf} n'est pas assez efficace. Nous avons donc implémenté
notre propre résolution de liens. Dans un second temps, nous avons continué à
optimiser en modifiant notre approche lors de la trace technique des objets
créés. Ces améliorations successives ont drastiquement diminué la durée
d'exécution de la seconde phase par rapport à la première qui concentre
maintenant l'essentiel du temps processeur d'une transformation.

Finalement, nous avons abouti à la version actuelle qui sera intégrée dans la
prochaine version stable de {\tom}. Tout au long de la transformation, nous
stockons les éléments tracés servant à la résolution (traçabilité technique)
ainsi que les éléments référençant les éléments intermédiaires \emph{resolve}
(c'est-à-dire ayant un pointeur vers un élément \emph{resolve}, par exemple un
arc image d'une \emph{WorkSequence} dans l'exemple
\emph{SimplePDLToPetriNet}). 

En début de phase de résolution, nous récupérons l'ensemble des éléments
intermédiaires temporaires. Nous le parcourons et pour chaque élément de cet
ensemble, nous parcourons l'ensemble des objets le référençant pour leur
appliquer la résolution de lien. Avec notre implémentation, nous parcourons le
minimum d'objets possible, ce qui réduit considérablement les parcours
(notamment par rapport à {\emf}). Ces optimisations nous ont apporté un gain
conséquent nous permettant de passer d'outils transformant difficilement des
petits modèles de \num{10000} éléments en des outils capables de transformer
des modèles de plusieurs millions d'éléments.

Finalement, ces premiers résultats expérimentaux nous permettent d'exprimer un
avis sur nos choix technologiques initiaux, en particulier sur l'utilisation de
{\emf}. Si ce choix paraissait naturel pour pouvoir avoir une couverture
initiale maximale de chaînes de développement, d'outils et d'utilisateurs, il
paraît beaucoup moins naturel lorsque l'on commence à s'intéresser aux
performances. En effet, notre travail de \emph{profiling} et d'optimisation
nous a montré que le facteur limitant essentiel résidait dans les appels
{\emf}. Leur limitation puis leur suppression a drastiquement amélioré les
performances de nos outils. Dans notre contexte, et pour les tâches demandées,
nous n'avions finalement pas un besoin fondamental de toutes les
fonctionnalités fournies par {\emf}. Une version simplifiée et optimisée de la
gestion des références inverses suffisait amplement pour notre phase de
résolution. Il est donc plus intéressant en termes de performances pures et de
mémoire de nous passer de la technologie de support ({\emf} dans notre
implémentation actuelle) pour tout ce qui relève de la mécanique interne de la
transformation. De plus, nous affranchir totalement d'une technologie nous
permet d'obtenir des outils plus génériques, et donc plus facilement
intégrables dans d'autres contextes et chaînes de développement.

Cependant, si le choix initial d'utilisation des services {\emf} pour le moteur
des transformations s'est avéré peu judicieux et a été remis en cause en cours
de thèse, il a été fondamental pour avoir une implémentation concrète et un
support de travail. De plus, notre travail a mis en avant ce choix peu
judicieux pour la mise en œuvre de mécanismes internes de nos transformations,
ce qui ne remet pas en cause l'usage d'{\emf} dans d'autres contextes avec
d'autres outils, ni même son usage avec {\tom}. En effet, au-delà des
constructions haut-niveau qui s'appuyaient partiellement sur {\emf}, subsiste
notre outil de génération automatique d'ancrages formels qui permet d'opérer le
changement d'espace technologique. 
%Il est totalement décorrélé du langage de transformation et de
%l'implémentation concrète de la phase de résolution. 
Cet outil reste extrêmement intéressant pour faciliter l'écriture des
transformations ---~de manière totalement indépendante du langage de
transformation ainsi que de sa mise en œuvre technique~--- dans un contexte
{\java}-{\emf}.


\section{Perspectives}
\label{ch:evaluation:perspectives}

%De notre travail de thèse ainsi que de ces premiers résultats expérimentaux,
%plusieurs pistes de recherche et de développement se profilent. 
L'ensemble de nos outils peut être actuellement considéré comme un prototype
opérationnel. Cependant, dans le cadre d'une utilisation plus large et dans un
contexte industriel réel, on pourrait s'orienter vers différents aspects qui
ont émergé de ce travail de thèse et des premiers résultats expérimentaux.

Nous avons pu éprouver l'intérêt de notre approche hybride pour développer des
transformations de modèles. Cette approche facilite de développement grâce aux
outils de génération de code et aux constructions haut-niveau avec une forte
expressivité, tout en conservant l'environnement global existant. Il reste donc
possible d'utiliser les spécificités du langage généraliste au sein duquel nous
nous intégrons pour certaines tâches tout en manipulant des termes dans une
autre partie de l'application. Le travail d'adaptation que l'utilisateur doit
fournir est de fait plus faible, d'où une plus grande efficacité.

Notre approche a aussi l'avantage de perturber au minimum la chaîne de
développement et le choix de l'adopter n'est pas irréversible ou coûteux dans
le sens où chaque transformation développée peut être remplacée par sa version
dans le langage hôte (ou écrite avec un autre outil) sans changer le reste du
logiciel. Dans cette optique d'intégration non-intrusive dans des
environnements existants, une voie serait de travailler à la généralisation de
nos outils afin d'étendre l'approche à d'autres langages et technologies. Par
exemple, un premier pas dans cette direction pourrait être de se pencher sur
l'utilisation de {\kevoree} ainsi que sur l'extension au langage {\ada} (des
travaux sur le sujet ont déjà été entamés).

L'utilisation de constructions haut-niveau et d'outils de génération permet à
l'utilisateur de se concentrer sur le cœur de l'application et de déléguer à un
outil automatique les tâches répétitives sujettes à erreurs pour un humain.
Cela contribue à augmenter la confiance dans le logiciel développé.

Un autre axe de travail consisterait à réfléchir sur les stratégies que nous
utilisons pour transformer des modèles.  Nous nous sommes aperçus que certaines
relations étaient souvent vues comme intéressantes (composition), mais que dans
certaines situations, d'autres relations pouvaient être le centre de la
transformation. Travailler sur la paramétrisation des stratégies de réécriture
par des types de relation à suivre pourrait être une voie à explorer.

Un autre aspect intéressant que d'autres outils couvrent est la transformation
de modèles multiples : actuellement notre approche considère un seul modèle
source et un seul modèle cible. Cependant, on pourrait gagner en expressivité
en permettant d'écrire des transformations à plusieurs sources et cibles. %Une
%première approche dans ce sens serait dans un premier temps de créer un modèle
%englobant ces différents modèles d'entrées

Bien que les outils soient complètement opérationnels, une optimisation serait
probablement nécessaire pour un usage plus large dans un cadre industriel, tant
du point de vue de leurs performances que de leur utilisabilité. Nos travaux
d'amélioration des performances par réingénierie de la phase de résolution
constituent une première étape qui sera concrétisée dans une prochaine
\emph{release} du projet {\tom}.

Du point de vue de la modularité et de la réutilisation du code, les stratégies
de réécriture nous fournissent un socle intéressant de par leur modularité
intrinsèque. Nos expérimentations sur le sujet ont abouti à un prototype de nos
outils avec phase de résolution modulaire. Cependant, pour des raisons de
confort d'utilisation pour le développeur, nous avons choisi de ne pas intégrer
cette fonctionnalité par défaut dans la version stable actuelle. Nous avons
cependant une base de travail pour réfléchir à cet aspect. Il faut toutefois
noter que les évolutions de nos outils dues à l'amélioration de leurs
performances risquent d'entrer en concurrence avec l'approche adoptée dans
notre prototype modulaire. De fait, l'aspect performance est actuellement
privilégié.

Nous avons aussi pu apporter une traçabilité au sein d'un langage généraliste
tel que {\java}. Il s'agissait d'apporter une aide substantielle au processus
de qualification d'un logiciel. L'aspect traçabilité des transformations semble
être un thème particulièrement prometteur. En effet, les industries développant
du logiciel critique ont un besoin accru de traçabilité pour qualifier et
certifier leurs outils. L'adoption de l'ingénierie dirigée par les modèles a
entraîné l'usage de nouvelles méthodes et de nouveaux outils tandis que les
contraintes légales subsistent. Partant de ce constat, les travaux sur la
traçabilité à des fins de validation du logiciel offrent des perspectives
intéressantes. Dans l'implémentation actuelle de nos travaux, les constructions
pour la traçabilité ne sont pas distinctes. Une première étape serait de
distinguer clairement (par deux constructions distinctes du langage) la
traçabilité utilisée à des fins purement techniques (usage pour la phase de
résolution) et la traçabilité métier. Notre approche de trace à la demande
permet déjà de limiter la taille des traces et donc de les rendre plus
exploitables \emph{a posteriori}. Se pose aussi la question de l'exploitation
du modèle de lien, de manière automatique ou non dans le processus de
qualification.



%virer ça

%\noindent \textbf{\textit{Approche hybride.}} Notre choix d'utiliser Tom est
%intéressant car nous nous plaçons à la frontière de deux espaces
%technologiques qui sont souvent peu connus ensembles par le développeur
%moyen. Cette approche nous permet d'utiliser des concepts et outils formels
%tout en restant accessible à des utilisateurs non-académiques. Il s'agit
%d'aider l'utilisateur à améliorer la qualité de son logiciel sans pour
%autant le pousser à changer radicalement d'environnement, d'outil, de langage
%et d'habitudes. Le cela permet de limiter le coût du développement par
%rapport à l'adoption d'outils complètement spécifiques et dédiés.
%
%\noindent \textbf{\textit{Langage de transformation.}} Concernant l'expression de la transformation elle-même ainsi que
%l'implémentation de l'approche en deux temps avec résolution, nous ne
%devrions pas apporter de changements fondamentaux. Il s'agit d'une approche
%relativement classique dans le monde des transformations de modèles qui a
%montré son efficacité. Les évolutions à venir viseront essentiellement
%l'amélioration de l'expérience utilisateur et la simplification de la prise
%en main des outils :
%\begin{itemize}
%\compresslist
%
%  \item L'introduction de raccourcis syntaxiques permettra de 
%%  L'écriture des définitions pourra être amélioré par l'introduction
%%  de raccourcis syntaxiques permettant de 
%  spécifier plus simplement la stratégie après le mot-clef
%  \lex{traversal}. Dans le cas le plus courant, l'utilisateur suit nos
%  recommandations et ne souhaite véritablement spécifier que le type de
%  parcours et non toute la stratégie (avec les paramètres la plupart du
%  temps identiques à ceux de la transformation) ;
%
%  \item Une simplification de l'utilisation du modèle de lien est aussi à
%  l'étude afin que l'utilisateur n'ait plus à le spécifier explicitement
%  ;
%
%  \item Nous souhaitons alléger la syntaxe de la construction \lex{\%resolve}
%  pour ne plus avoir à donner explicitement les types en les inférant.
%
%\end{itemize}
%
%\noindent \textbf{\textit{\'Evaluation et limitations.}} Nous avons essayé
%de trouver des usages pour lesquels nos outils étaient particulièrement
%intéressants ou au contraires pour lesquels ils apportaient peu. Nous les
%avons notamment utilisés pour implémenter l'exemple de l'applatissement
%d'une hiérarchie de classes (les feuilles de l'arbre d'héritage
%récupèrent tous les attributs de leurs surclasses).  Cet exemple nous
%semblait intéressant du fait que la relation de composition n'est pas
%centrale et que notre outil est calibré pour la relation de composition. \\
%Nous avons constaté que pour un tel exemple dont l'implémentation
%récursive en Java, Java+Tom classique (sans la partie modèle) ou ATL est
%triviale, les constructions dédiées aux modèles n'apportent aucun
%véritable gain. Elles complexifient même la transformation, sans
%véritablement tirer parti des outils développés. Nous constatons aussi
%que \lex{\%transformation} prend beaucoup d'intérêt lors de l'utilisation
%d'éléments \emph{resolve}, car les structures de données ainsi que la
%phase de résolution sont générées automatiquement. En revanche, cet
%exemple n'en nécessitant pas, le code est plus concis et lisible avec du Tom
%classique sans construction \lex{\%transformation}. Cet exemple nous a aussi
%amené à réfléchir à l'extension de l'\emph{introspecteur} à
%d'autres liens que les liens de composition (paramétrisation par les liens).
%
%\noindent \textbf{\textit{Traçabilité.}} Ce faible intérêt apparent
%pour les transformations les moins complexes peut être compensé par un
%autre axe de travail actuel : la traçabilité des transformations. Elle
%est actuellement assurée par le modèle de lien et la construction
%\lex{\%tracelink}, cependant cela comprend deux aspects de la traçabilité que
%nous souhaitons distinguer : la traçabilité
%\emph{technique} et de \emph{spécification}.\\
%La traçabilité \emph{technique} est la traçabilité utilisée en
%interne pour la transformation elle-même (pour la phase de résolution).
%Bien qu'extrêment utile à des fins de debug, Cette traçabilité
%proche de l'implémentation peut être très éloignée d'une
%spécification. La certification d'un logiciel peut par conséquent être
%particulièrement ardue avec cette unique traçabilité, qui est
%d'ailleurs implémentée dans ATL ~\cite{Jouault2005} et
%Kermeta~\cite{Falleri2006}.\\
%La traçabilité de \emph{spécification} peut quant à elle être
%décorrélée de l'implémentation technique. Elle décrit des relations
%entre des sources et des cibles comme une spécification peut le faire, sans
%forcément adopter le découpage des règles de l'implémentation. Nous
%souhaitons séparer clairement ces deux traçabilités et donc découper
%la construction \lex{\%tracelink} en deux constructions distinctes qui joueront
%chacune un r\^ole spécifique. Techniquement, il nous faudra lever la
%limitation actuelle de ne pouvoir lier qu'une seule source à plusieurs
%cibles. Cela nous permettra de travailler à la généralisation de la
%traçabilité d'une transformation pour les approches hybrides, et à
%l'ajout de la traçabilité dans les transformations écrites avec des
%langages généralistes grand public.
%
%\noindent \textbf{\textit{Enrichissement de l'expressivité du langage.}}
%Notre solution est proche de celle apportée par ATL. Cependant, nous nous
%distonguons par notre volonté de nous intégrer totalement à des langages
%généralistes (et particulièrement Java). Nous souhaitons aussi étendre
%et généraliser le mécanisme \emph{resolve} à des sources multiples, ce
%que ne permet pas ATL.  Actuellement, un élément \emph{resolve} (ou le
%\emph{resolveTemp} d'ATL) ne lie qu'un élément source à un élément
%cible. Nous souhaitons améliorer l'expressivité de notre langage en offrant
%la possibilité du \emph{resolve} multi-sources ainsi que des règles
%multi-\emph{patterns}. Cela passe par l'élaboration de nouveaux mécanismes
%tels que des \emph{patterns} paramétrés ou des stratégies multi-sujets
%sur lesquelles nous travaillons actuellement.
%
%\noindent \textbf{\textit{Modularité.}} Ensuite, dans l'optique d'accroître
%la modularité de notre approche -- et donc la réutilisabilité des
%transformations --, un nouvel axe de travail est de décomposer la phase de
%résolution jusqu'alors monolithique en \emph{résolutions élémentaires}.
%Si une définition pouvait être réutilisée dans une autre
%transformation, il était nécessaire de générer (ou éventuellement
%développer) une nouvelle phase de résolution adaptée à la nouvelle
%transformation. En décomposant cette phase en briques élémentaires, nous
%rendons la phase de résolution modulaire et réutilisable. Une définition
%-- accompagnée des résolutions élémentaires associées -- pourra donc
%être complètement portable dans d'autres transformations.
%
%\noindent \textbf{\textit{Usage industriel.}} Bien que les versions courantes
%du langage et des outils puissent être encore améliorées, l'ensemble est
%déjà utilisable dans un contexte non-académique. Nos outils ont été
%utilisés par nos partenaires industriels durant le projet
%Quarteft~\footnote{\url{http://www.quarteft.loria.fr}} financé par la
%FRAE~\footnote{Fondation de Recherche pour l'Aéronautique et l'Espace :
%\url{http://www.fnrae.org}}. 

% vim:spell spelllang=fr
