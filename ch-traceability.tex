% vim:spell spelllang=fr
\chapter{Spécification et traçabilité des transformations}
\label{ch:traceability}
%10p

Dans ce chapitre, nous abordons la notion de spécification et de traçabilité
d'une transformation de modèles.
%Dans ce chapitre, nous expliquons ce qu'est la traçabilité et comment spécifier
%celle d'une transformation de modèle.

Les systèmes se complexifiant, l'{\idm} a apporté des solutions pour faciliter
et accélérer le développement logiciel. Cependant, si la manière de développer
un logiciel ainsi que les technologies utilisées ont changé depuis les débuts
de l'informatique, les contraintes liées à la maintenance (évolution du
logiciel, déboguage) et à la fiabilité (qualification, certification)
restent identiques. 
%Ainsi, dans le cadre de la qualification, il est nécessaire
%d'adopter la traçabilité. Lorsque l'on écrit et utilise une transformation, il
%est nécessaire de connaître et de conserver les relations entre les sources et
%les cibles. Ces relations sont données par la trace.
Ainsi, dans le cadre des transformations qualifiables, nous souhaitons avoir
confiance dans les transformations développées, du point de vue formel (pour la
vérification) ainsi que de celui de l'autorité de certification. La norme
DO-178/ED-12 exige la traçabilité entre le code source et le code objet, c'est
donc une problématique essentielle des compilateurs dans le contexte de la
qualification. Cette problématique est généralisable à toute transformation,
notamment aux transformations de modèles. En effet, l'ingénierie dirigée par
les modèles étant de plus en plus présente dans les chaînes de développement de
systèmes critiques, du code généré à partir d'un modèle peut faire partie du
logiciel final.



\section{Spécification}

Dans le chapitre~\ref{ch:verification}, nous avons vu qu'il fallait qualifier
les transformations dans le cadre du développement de systèmes critiques. Il
est important de spécifier les transformations pour vérifier leur conformité,
afin de disposer de la traçabilité code source--code objet. Pour cela, nous
nous appuyons sur les transformations de modèles. %\todo{La traçabilité s'effectue au
%niveau de la spécification, or les langages tels que {\qvtr} ou {\atl}
%proposent une traçabilité mécanique dans l'implémentation. De manière générale,
%cette traçabilité technique n'est pas substituable à celle de spécification. }

Reprenons le cas d'utilisation \emph{SimplePDLToPetriNet} brièvement illustré
dans le chapitre précédent et spécifions les modèles source et destination
ci-après.

\begin{figure}[h]%[fig:simplepdlmmodel]{Métamodèle SimplePDL.}%[H] %[!h]
  \begin{center}
    %\begin{tikzpicture}[scale=0.61,transform shape]
\begin{tikzpicture}%[scale=1,transform shape]

  \begin{class}{Process}{-8,3}
    \attribute{name : String}
  \end{class}
  
  \begin{abstractclass}{ProcessElement}{0,3}
  \end{abstractclass}
  
%  \begin{class}{Guidance}{-1,0.5}
%    \inherit{ProcessElement}
%    \attribute{text : String}
%  \end{class}
  
  \begin{class}{WorkDefinition}{-8,-1.5}
    \inherit{ProcessElement}
    \attribute{name : String}
  \end{class}
  
  \begin{class}{WorkSequence}{0,-1.5}
    \inherit{ProcessElement}
    \attribute{linkType : WorkSequenceType}
  \end{class}
  
  \begin{enum}{WorkSequenceType}{2.5,1.5}
    \attribute{startToStart}
    \attribute{finishToStart}
    \attribute{startToFinish}
    \attribute{finishToFinish}
  \end{enum}
  
  \composition{Process}{processElements}{*}{ProcessElement}
  %\composition{WorkDefinition}{process}{0..1}{Process}

  %\unidirectionalAssociation{Process}{from}{0..1}{WorkDefinition}
  \unidirectionalAssociation{ProcessElement}{parent}{1}{Process}

  %\association{WorkDefinition}{successor}{1}{WorkSequence}{0..*}{linksToPredecessors}
  \myassociation{WorkDefinition}{successor}{1}{WorkSequence}{*}{linksToPredecessors}{-4.5,-1}{0} %{210}

  %\association{WorkDefinition}{predecessor}{1}{WorkSequence}{0..*}{linksToSuccessors}
  \myassociation{WorkDefinition}{predecessor}{1}{WorkSequence}{*}{linksToSuccessors}{-4.8,-3.2}{1}  %{150}

\end{tikzpicture}

    \caption{Métamodèle SimplePDL possible.}
    \label{fig:simplesimplepdlmmodel}
  \end{center}
\end{figure}

Le langage SimplePDL dont le métamodèle est donné
figure~\ref{fig:simplepdlmmodel} permet d'exprimer simplement des processus
génériques. Un processus (\emph{Process}) est composé d'éléments
(\emph{ProcessElement}). Chaque \emph{ProcessElement} référence son processus
\emph{parent} et peut être soit une \emph{WorkDefinition}, soit une
\emph{WorkSequence}. Une \emph{WorkDefinition} définit une activité qui doit
être effectuée durant le processus (un calcul, une action, {\etc}). Une
\emph{WorkSequence} définit quant à elle une relation de dépendance entre deux
activités. La deuxième (\emph{successor}) peut être démarrée ---~ou
terminée~--- uniquement lorsque la première (\emph{predecessor}) est déjà
démarrée ---~ou terminée~--- selon la valeur de l'attribut \emph{linkType} qui
peut donc prendre quatre valeurs : \emph{startToStart}, \emph{finishToStart},
\emph{startToFinish} ou \emph{finishToFinish}. Afin de pouvoir représenter des
processus hiérarchiques, une \emph{WorkDefinition} peut elle-même être définie
par un processus imbriqué (référence \emph{process}), qui conserve un lien vers
l'activité qu'il décrit (référence \emph{from}).

Le métamodèle donné par la figure~\ref{fig:petrinetmmodel2} permet d'exprimer
les réseaux de Petri. Un tel réseau se définit par un ensemble de nœuds
(\emph{Node}) qui sont soit des places de type \emph{Place}, soit des
transitions de type \emph{Transition}, ainsi que par des arcs (\emph{Arc}). Un
arc (orienté) relie deux nœuds de types différents (le réseau de Petri est un
graphe biparti) et peut être de type \emph{normal} ou \emph{read-arc}. Il
spécifie le nombre de jetons (\emph{weight} ---~poids~---) consommés dans la
place source ou produits dans la place cible lorsqu'une transition est tirée.
Un \emph{read-arc} vérifie uniquement la disponibilité des jetons sans pour
autant les consommer (test de franchissement). Le marquage d'un réseau de Petri
est défini par le nombre de jetons dans chaque place (\emph{marking}).

\begin{figure}[h]%[fig:petrinetmmodel]{Métamodèle PetriNet.}%[H] %[!h]
  \begin{center}
    %\begin{tikzpicture}[scale=0.7,transform shape]
\begin{tikzpicture}[scale=1.0,transform shape]

  \begin{class}{PetriNet}{2,-0.5}
    \attribute{name : String}
  \end{class}
  
  \begin{class}[text width=2.5cm]{Arc}{1,3}
    \attribute{kind : ArcKind}
    \attribute{weight : Int}
  \end{class}
  
  \begin{abstractclass}{Node}{-5,0.7}
    \attribute{name : String}
  \end{abstractclass}
  
  \begin{class}{Place}{-2,-1}
    \inherit{Node}
    \attribute{marking : Int}
  \end{class}
  
  \begin{class}{Transition}{-8,-1}
    \inherit{Node}
    %\attribute{min\_time : Int}
    %\attribute{max\_time : Int}
  \end{class}
  
  \begin{enum}{ArcKind}{-7,3}
    \attribute{normal}
    \attribute{read\_arc}
  \end{enum}
  
  \composition{PetriNet}{nodes}{*}{Node}
  \unidirectionalAssociation{Node}{net}{1}{PetriNet}
  
  \composition{PetriNet}{arcs}{*}{Arc}
  \unidirectionalAssociation{Arc}{net}{1}{PetriNet}

  %\association{Arc}{source}{1}{Node}{0..*}{outgoings}
  \myassociation{Node}{source}{1}{Arc}{*}{outgoings}{-3,2.1}{0}

  %\association{Arc}{target}{1}{Node}{0..*}{incomings}
  \myassociation{Node}{target}{1}{Arc}{*}{incomings}{-2,0.8}{1}

\end{tikzpicture}

    \caption{Métamodèle des réseaux de Petri.}
    \label{fig:petrinetmmodel2}
  \end{center}
\end{figure}

\FloatBarrier

\section{Traçabilité}

La traçabilité est très importante dans le cadre de la maintenance pour pouvoir
suivre l'évolution d'un logiciel et détecter des \emph{bugs} introduits durant
le cycle de vie de l'application. C'est un précieux outil pour l'analyse et
la vérification de transformations, ce qui explique que ce soit une exigence de
qualification. La traçabilité peut prendre plusieurs formes. Nous distinguons
notamment la traçabilité \emph{interne} (autrement appelée \emph{technique}) de
la traçabilité de spécification.

\subsection{Traçabilité interne}

La traçabilité \emph{interne} ou \emph{technique}~\cite{Jouault2005} a un usage
purement technique au sein de la transformation. C'est-à-dire qu'elle est
utilisée pour opérer la transformation, mais n'est pas nécessairement utile dans
le cadre d'un processus de qualification. Dans les approches compositionnelles
des transformations comme la nôtre, il faut mettre en œuvre un mécanisme
permettant d'assembler les résultats partiels. Dans notre cas, si des éléments
intermédiaires \emph{resolve} sont créés, il est nécessaire de pouvoir les
faire correspondre à des éléments cibles réels. Il faut donc pouvoir marquer
les éléments susceptibles de correspondre à un \emph{resolve} donné afin
d'assurer la réalisation de la phase de résolution. Nous avons mis en œuvre
cette traçabilité dans le cadre de notre approche décrite dans le
chapitre~\ref{ch:approach} et nous la décrivons plus précisément dans le
chapitre~\ref{ch:outils}. Nous ne nous attardons donc pas sur cette notion dans
ce chapitre.

Du fait de son usage, la traçabilité technique est étroitement liée à
l'implémentation de la transformation. Elle est difficilement générique, ce qui
peut nécessiter de la réingénierie lors d'une évolution de la transformation.

%\todo{Deux types de traçabilité pouvant être distingués, nous nous concentrons
%essentiellement sur la
%traçabilité de spécification dans ce chapitre. En effet, la traçabilité
%\emph{interne} (ou \emph{technique}) ---~purement mécanique~--- est liée
%%permet de réaliser le 
%mécanisme de résolution exposé dans le chapitre précédemment.}

%\todo{types de traçabiilité : implicite/explicite ; interne/spec ; choix ;
%approche}


\subsection{Traçabilité de spécification}

La traçabilité de spécification est quant à elle une exigence de qualification.
C'est-à-dire qu'il faut pouvoir lier une source à une cible selon une
spécification donnée. Ce type de traçabilité n'a pas d'usage technique pour le
bon déroulement de la transformation elle-même et n'est pas nécessairement liée
à l'implémentation.

La traçabilité des transformations de modèles étant un élément essentiel dans
un cadre industriel, beaucoup de langages l'ont implémenté. Cependant, les
langages tels que {\qvtr} et {\atl} proposent une traçabilité mécanique, mais
qui n'est pas substituable à la traçabilité de spécification.


%\ttodo{==============================
%  \begin{itemize}
%    \item traçabilité vs spec
%    \item intérêt
%    \item que tracer ? quel format ? usage
%    \item spec MM + explication
%    \item spec textuelle
%    \item OCL contrainte sur la spec textuelle
%    \item approche implicite
%  \end{itemize}
%       ==============================}


En {\idm}, l'usage de {\uml} est courant et comme nous avons pu le voir dans le
chapitre~\ref{ch:notions}, un métamodèle permet de spécifier un langage. Pour
exprimer la traçabilité dans ce contexte, il est donc naturel d'utiliser des
métaclasses pour avoir un formalisme cohérent. On peut alors exprimer
simplement la traçabilité par le métamodèle~\ref{fig:linkmmodel}.

\begin{figure}[h]
  \begin{center}
    %\begin{tikzpicture}[scale=0.7,transform shape]
\begin{tikzpicture}[scale=1,transform shape]

  \begin{class}{Trace}{-5,0}
    \attribute{transfoName : String}
  \end{class}
  
  \begin{class}{TraceLink}{0,0}
    \attribute{ruleName : String}
  \end{class}
  
  \begin{class}{SourceElement}{4.5,1}
    \attribute{}
  \end{class}
 
  \begin{class}{TargetElement}{4.5,-1}
    \attribute{}
  \end{class}
  
  \composition{Trace}{links}{*}{TraceLink}
  \draw [umlcd style, ->] ([yshift=40pt]TraceLink) -- (SourceElement)
  node[near end, above,xshift=-12pt]{sources 1..*};
  %node[near end, above]{sources}
  %node[near end, below]{*};

\draw [umlcd style, ->] ([yshift=-40pt]TraceLink) -- (TargetElement)
  node[near end, below,xshift=-12pt]{targets 1..*};
%  node[near end, below]{targets}
%  node[near end, above]{*};
%controls +(-160:5em) and +(140:5em) ..

%  \unidirectionalAssociation{TraceLink}{sources}{*}{Element}
%  \unidirectionalAssociation{TraceLink}{targets}{*}{Element}
  
\end{tikzpicture}

    \caption{Métamodèle générique de trace.}
    \label{fig:linkmmodel}
  \end{center}
\end{figure}

Une \emph{Trace} concerne une transformation donnée nommée. Elle est composée
d'un ensemble de relations (ou de liens de trace \emph{TraceLink}) entre des
éléments sources et cibles. Un lien de trace établit une relation entre un ou
plusieurs éléments sources (\emph{SourceElement}) et un ou plusieurs éléments
cibles (\emph{TargetElement}). Une relation de trace est nommée et peut être
considérée comme une règle ayant un membre gauche (source) et en membre droit
(cible). Ce métamodèle est générique afin de donner une intuition. Il doit
cependant être spécifique pour chaque transformation donnée. Dans le cas de
\emph{SimplePDLToPetrinet}, \emph{SourceElement} de la
figure~\ref{fig:linkmmodel} est remplacé par les métaclasses \emph{Process},
\emph{WorkDefinition} et \emph{WorkSequence} ou alors est leur surtype. Dans le
cas d'un \emph{framework} comme {\emf}, on peut utiliser \emph{EObject}.

Lors de la certification et de la qualification, les spécifications sont
cependant très souvent écrites en langue naturelle (français ou anglais). Le
tâche de l'expert qualifieur consiste alors à comparer la spécification
textuelle au code du logiciel et aux sorties produites pour des entrées
données. Cette tâche est complexe et fortement sujette à erreurs du fait du
rôle central que joue l'humain dans ce processus. Tout doit donc être mis en
œuvre pour assister l'expert qualifieur en lui fournissant des éléments de
confiance supplémentaires. C'est le rôle des outils et de leurs implémentations
de la traçabilité.

Cette trace peut revêtir différents aspects et n'est pas forcément utilisable
avec des outils automatiques. Par exemple, l'implémentation de la traçabilité
peut consister en la génération de commentaires lisibles par un humain.  C'est
d'ailleurs l'approche adoptée par l'outil {\btollvm} ---~hors du contexte de
l'{\idm}~--- qui opère une transformation du langage {\B} vers le langage
intermédiaire de {\llvm}\footnote{Le projet {\llvm} est une infrastructure de
compilateur modulaire et réutilisable, voir \url{http://www.llvm.org}.}. Ces
commentaires facilitent la lecture et la compréhension du code par l'expert,
mais ne sont cependant pas réutilisables en tant que tels comme entrées d'un
outil automatique de vérification. Un point particulièrement intéressant est
d'assurer la traçabilité d'une transformation en générant une trace formelle
exploitable \emph{a posteriori} par d'autres outils. L'intérêt est alors de
permettre de se passer de l'écriture d'une preuve formelle ---~coûteuse à
obtenir~--- tout en conservant une confiance forte dans la transformation. Une
telle trace générée lors d'une transformation peut-être utilisée de deux
manières :
\begin{itemize}
  \item elle peut être comparée à une trace de référence donnée en spécification ;
  \item des propriétés données en spécification peuvent être vérifiées.
\end{itemize}

Dans le premier cas d'utilisation, le format de la trace doit être parfaitement
spécifié afin que, pour une entrée et une transformation données, la trace
puisse être comparée à une trace de référence. Cette dernière fait alors office
de spécification de la transformation. 

Dans le second cas d'utilisation de la trace, il faut pouvoir exprimer des
propriétés à vérifier. Dans le contexte de la modélisation, {\ocl} est utilisé
pour décrire des règles s'appliquant sur des modèles {\uml}. Lors d'une
transformation de modèle, des pre-conditions peuvent donc être écrites pour les
éléments sources, des post-conditions pour les éléments cibles et des
invariants pour les liens de traçabilité.

Dans notre approche, nous avons proposé une traçabilité de spécification. La
transformation par réécriture présentée étant une transformation par parties,
nous agrégeons un ensemble de règles (\emph{définitions}) pour obtenir la
transformation finale. Chacune de ces \emph{définitions} est nommée et nous
utilisons son nom pour construire le lien de trace lors de la génération du
métamodèle de lien (au temps de compilation). Le membre gauche de nos règles
constitue la source de chaque lien de trace, tandis que les éléments du membre
droit sont les cibles de la relation de trace. Sur action explicite de
l'utilisateur, nous établissons une relation un élément source et élément
cible. Dans sa forme actuelle, notre implémentation permet de lier plusieurs
cibles à une seule source (relations \texttt{1..N}), l'objectif étant à terme
d'établir des liens entre plusieurs sources et plusieurs cibles (relations
\texttt{N..N}).

Reprenons le cas de la transformation \emph{SimplePDLToPetriNet} et choisissons
une règle comme exemple : \emph{Process2PetriNet}. Cette règle transforme les
éléments \emph{Process} du modèle source en un réseau de Petri composé de trois
places ($p\_ready$, $p\_running$ et $p\_finished$), deux transitions
($t\_start$ et $t\_finish$) et de quatre arcs ($ready2start$, $start2running$,
$running2finish$ et $finish2finished$). De cette description textuelle, nous
pouvons spécifier la relation liant la source \emph{src} aux éléments cibles
comme suit :
\begin{verbatim}
Process2PetriNet = {
  src : Process ;
  p_ready, p_running, p_finished : Place ;
  t_start, t_finish : Transition ;
  ready2start, start2running, running2finish, finish2finished : Arc
}
\end{verbatim}

On peut faire de même avec les relations \emph{WorkDefinition2PetriNet} et
\emph{WorkSequence2PetriNet} pour compléter la spécification.

Il est ensuite possible d'exprimer des contraintes sur cette relation,
notamment des contraintes de nommage des éléments cibles. Dans notre cas,
supposons que l'on impose que le nom d'un élément cible soit composé de deux
parties, sur le principe suivant :
\begin{itemize}
  \item le nom de l'élément source lui ayant donné naissance en préfixe ;
  \item un suffixe pertinent pour décrire l'élément (par exemple \emph{ready}
    pour la place \emph{p\_ready}).
\end{itemize}

Avec ces conventions, nous pouvons par exemple écrire la contrainte de nommage
pour la place \emph{p\_ready} peut être écrite comme suit (\verb+concat()+ est
l'opération de concaténation des chaînes de caractères dans {\ocl}) :
\begin{verbatim}
p_ready.Name = src.Name.concat('_').concat(p_ready.Name);
\end{verbatim}

La trace est donc exploitable à des fins de qualification.

%L'un des intérêts de
%l'{\idm} est d'avoir la capacité de fournir des 

%Compte tenu du processus de qualification actuel et de l'importance de cette 

%Cette problématique étant fondamentale pour tout développement de système
%critique, dans un autre domaine que l'{\idm}, 

%C'est d'ailleurs l'approche adoptée par l'outil {\btollvm} qui implémente la
%traçabilité de la transformation du langage {\B} vers le langage intermédiaire
%de {\llvm} sous la forme de commentaires lisibles par un humain. Ces
%commentaires sont générés pour un code {\B} donné et évoluent donc avec
%l'outil. Ils facilitent la lecture et la compréhension du code par l'expert
%mais ne sont pas réutilisables en tant que tel comme entrée d'un outil
%automatique de vérification.


%(\emph{SourceElement}) ---~association \emph{sources}~--- et un ou plusieurs
%éléments cibles (\emph{TargetElement}) --~association \emph{targets}~---.
%\ttodo{là il y a un problème : où est le liant ? d'où est-ce que ça sort ? etc.}

%Ensuite, on se propose d'assurer la correction en exprimant les
%propriétés avec {\ocl}. 
%On écrit des pre-conditions pour les sources, des
%post-conditions pour les cibles et des invariants pour les liens de
%traçabilité. Ainsi, dans notre exemple, la traçabilité peut être spécifiée sous
%la forme de contraintes {\ocl} telles que \todo{Nom du \emph{Process}
%SimplePDL = Préfixe des éléments du réseau de Petri image ; Nom d'une
%\emph{WorkDefinition} = Préfixe des éléments du réseau de Petri image de
%l'activité}. (\ttodo{en OCL ?}).


%Un modèle = nœuds + arcs (seront attribués)
%
%capturer 1 certain nombre de nœuds et d'arcs $\Rightarrow nouvel ensemble$
%
%nœuds bidons (resolve)
%
%Transfo de Modèles : {nœuds, arcs} -> autre {n,a} + nœuds à trous (resolve)
%
%Transfo pour réécriture de graphe
%
%Comment construire cette fonction ? par parties (bouts de graphes, avec des
%trous)
%
%réécriture de termes $\Rightarrow$ graphe + arbre de recouvrement (par la
%composition) + liens (associations)
%%
%%mécanique de termes, RW arbres
%
%+ enrichissement asso + recoller les morceaux de graphes
%%
%%description compositionnelle du problème : règles, assemblage via resolve.

%\ttodo{ }

%\ttodo{tout ce qui suit doit disparaitre d'une manière ou d'une autre ============}

%\ttodo{réorganisation complète :
%  \begin{itemize}
%%    \item Comment spécifier ? illustrer la traçabilité (sans technique)
%%    \item spéc sous forme de contraintes OCL : Nom petri = Nom process, etc.
%%    \item propriétés OCL pre src / post tgt / inv lines de traçabilité
%    \item cas d'utilisation SimplePDL2PetriNet à remonter du ch7 (MM) (modèles
%      simples dans le ch4)
%    \item modèle = ensemble de nœuds et arcs
%    \item transfo de graphe, autre ensemble avec en intermédiaire la possibilité d'introduire des nœuds resolve
%    \item transfo pour réécrire le graphe
%    \item comment construire cette fonction ? par parties (bouts de graphe,
%      éventuellement des bouts avec des trous)
%    \item réécriture de termes, graphe -> arbre de recouvrement (composition) +
%      liens (association)
%  \end{itemize}
%}

%1e point : cf ch3, système critiques, qualification de transfo\\
%important de spécifier la transformation pour vérifier la conformité
%disposer de la traçabilité CS-CO\\

%$\Rightarrow$ s'appuyer pour transfo les modèles\\
%niveau spéc, donc différent de QVTR\\
%ex comment spécifier
%
%ch5, comment spécifier $\Rightarrow$ illustrer traçabilité (sans tech)
%
%spéc sous forme de contraintes OCL : nom Petri = nom Process (par exemple), nom
%activité

%si on génère les liens de traçabilité, contraintes OCL applicables $\Rightarrow$
%confiance$+$ $\Rightarrow$ pas besoin de preuve

%2e point : langage QVTR, ATL, etc. proposent une traçabilité mécanique de
%l'implémentation, pas substituable à celle de spécification

%ex : aplatissement machines à états

%\ttodo{tout ce qui précède doit disparaître d'une manière ou d'une autre\\============}
%\ttodo{=========================================}


%Metamodèle dans la vraie vie : 
%\begin{figure}[h]
%  \begin{center}
%    %\begin{tikzpicture}[scale=0.7,transform shape]
\begin{tikzpicture}[scale=1,transform shape]

  \begin{class}{Trace}{-5.5,0}
    \attribute{transfoName : String}
  \end{class}
  
  \begin{class}{ReferenceClass}{0,0}
    \attribute{definitionName : String}
  \end{class}
 
  \begin{class}{SourceElement}{5.0,1}
    \attribute{}
  \end{class}
 
  \begin{class}{TargetElement}{5.0,-1}
    \attribute{}
  \end{class}
  
%  \begin{class}{Element}{4.5,0}
%    \attribute{}
%  \end{class}
%  
%  \composition{Trace}{links}{*}{TraceLink}
%  \draw [umlcd style, ->] ([yshift=40pt]TraceLink) -- (Element)
%  node[near end, above,xshift=-7pt]{source 1};
%  %node[near end, above]{sources}
%  %node[near end, below]{*};
%
%\draw [umlcd style, ->] ([yshift=-40pt]TraceLink) -- (Element)
%  node[near end, below,xshift=-7pt]{targets *};
%%  node[near end, below]{targets}
%%  node[near end, above]{*};
%%controls +(-160:5em) and +(140:5em) ..
%
%%  \unidirectionalAssociation{TraceLink}{sources}{*}{Element}
%%  \unidirectionalAssociation{TraceLink}{targets}{*}{Element}

  %test
  \composition{Trace}{links}{*}{ReferenceClass}
  \draw [umlcd style, ->] ([yshift=40pt]ReferenceClass) -- (SourceElement)
  node[near end, above,xshift=-14pt]{sources 1..*};
  %node[near end, above]{sources}
  %node[near end, below]{*};

\draw [umlcd style, ->] ([yshift=-40pt]ReferenceClass) -- (TargetElement)
  node[near end, below,xshift=-14pt]{targets 1..*};



\end{tikzpicture}

%    \caption{Métamodèle de trace généré}
%    \label{fig:actuallinkmmodel}
%  \end{center}
%\end{figure}


%\todo{todo: 
%\begin{itemize}
%  \item Intérêt
%  \item types
%  \begin{itemize}
%    \item interne
%    \begin{itemize}
%      \item explication
%      \item implémentation technique NON
%    \end{itemize}
%    \item spécification
%    \begin{itemize}
%      \item explication
%      \item implémentation technique / projet d'implémentation NON
%    \end{itemize}
%  \end{itemize}
%\end{itemize}}

%\ttodo{??? \begin{itemize}
%\item DO-330 <- exigences dans l'industrie, pour la qualification du logiciel
%\item que tracer ? comment ? quel format
%\item intérêt
%\item une idée de modèle de lien
%\item comment utiliser les traces ? avec quels outils ?
%\end{itemize}}

%Les systèmes se complexifiant, l'{\idm} a apporté des solutions pour faciliter
%et accélérer le développement logiciel. Cependant, si la manière de développer
%un logiciel ainsi que les technologies utilisées ont changé depuis les débuts
%de l'informatique, les contraintes liées à la maintenance (évolution du
%logiciel, déboguage) ainsi qu'à la fiabilité (qualification, certification)
%restent identiques. Ainsi, dans le cadre de la qualification, il est nécessaire
%d'adopter la traçabilité. Lorsque l'on écrit et utilise une transformation, il
%est nécessaire de connaître et de conserver les relations entre les sources et
%les cibles. Ces relations sont données par la trace.

%Nous abordons la notion de traçabilité

%La notion de traçabilité est fondamentale en {\idm} pour différentes raisons
%: %logiciel construit à partir de modèles, puis génération de code =>
%%maintenance difficile
%\begin{itemize}
%\item qualification de logiciels obtenus via IDM
%\item maintenance logicielle : changement d'archi, détection etcorrection de
%bugs, etc.
%\item analyse de transformations/compilations
%\end{itemize}

%définition de la \emph{trace navigation}~\cite{Vanhooff}


%\section{Traçabilité interne}
%%\subsection{\todo{Explication}}
%usage pour les transfos elles-mêmes ; fortement liée à l'implémentation ;
%problèmes en cas de réingénierie.


%\subsection{Implémentation technique}
%la partie « resolvelink » de \lex{\%tracelink} ; faut vraiment découper la
%  construction en deux pour que les deux traçabilités soient bien
%  distinctes 


%\section{Traçabilité des spécifications}
%%\subsection{\todo{Explication}}
%intérêt ; découpler la trace des règles implémentées. La traçabilité de
%spécification n'est pas un debug ; 
%
%%\subsection{Mise en {\oe}uvre / projet d'implémentation}
%sur le mode « tout est modèle », je veux « tout est terme ». Le \emph{modèle de
%lien} devient un \emph{terme de lien}.\\
%TRACE, qui est l'ouverture vers un « contexte/environnement de termes ». 

\section{Synthèse}%Approche}
%\ttodo{en vrac :  environnement hybride / 1..N / marquage /}

Nous avons vu qu'on pouvait distinguer deux types de traçabilité :
\emph{interne} et \emph{de spécification}. Nous avons mis en œuvre les deux, la
première étant absolument nécessaire à la bonne réalisation de notre approche,
la seconde étant utile pour le développement de transformations qualifiables.

Nous avons vu dans le chapitre~\ref{ch:verification:trace} que la traçabilité
est généralement classée en deux catégories : \emph{implicite} et
\emph{explicite}. Un aspect important de notre approche réside dans notre choix
d'une traçabilité \emph{explicite}, ce qui implique l'introduction d'une
syntaxe concrète dans le langage pour capturer explicitement les liens de
trace. Dans le cas d'un choix implicite, tout doit être tracé et il faut
prévoir un mécanisme pour trier et sélectionner les informations pertinentes
\emph{a posteriori} (un système de requête sur la trace générée par exemple).
Notre choix présente l'intérêt de produire des traces ciblées d'une taille plus
raisonnable. Elles sont donc exploitables plus facilement \emph{a posteriori}.

Enfin, un autre point intéressant qui ne peut transparaître sans aborder la
question technique tient au fait de notre environnement technologique  hybride.
En effet, nous apportons une traçabilité au sein d'un langage généraliste par
l'ajout d'une construction dédiée. Dans un environnement homogène et parfaitement maîtrisé, la
difficulté est levée rapidement, la traçabilité pouvant être assurée
relativement simplement. En revanche, dans notre contexte, nous sommes à la
frontière de deux mondes que l'utilisateur peut franchir selon ses convenances.
Si l'utilisation du langage {\tom} et la manipulation de termes ne lui
conviennent pas, l'utilisateur peut revenir à un mode de programmation en pur
{\java}, auquel cas nous perdons le contrôle de cette partie du code. Ajouter
la traçabilité dans cet environnement devient plus complexe. Nous détaillons la
mise en œuvre de notre approche dans le chapitre~\ref{ch:outils}.

Finalement, notre approche permet un usage \emph{a posteriori} de la trace
générée à des fins de vérification dans le cadre de la qualification.

%\todo{tout ce qui suit doit disparaitre ============================== }
%
%Notre approche sur la traçabilité d'une transformation consiste à exprimer
%explicitement les relations liant une source à une cible. Nous sommes
%actuellement en mesure d'associer une source à plusieurs cibles (relations
%\texttt{1..N}). Nous disposons des deux types de traçabilité : d'une part la
%traçabilité \emph{interne} (ou \emph{technique}) qui nous permet d'assurer la
%phase de résolution, d'autre part la traçabilité des spécifications. Notre
%approche est intéressante dans le sens où la granularité de la trace n'est pas
%imposée à l'utilisateur (trop haut niveau ---~modèle source lié au modèle
%cible~--- ou trop bas niveau ---~tous les éléments cibles sont forcément liés à
%des éléments sources~---). Un autre intérêt certain de notre approche est son
%environnement technologique : nous nous plaçons à la frontière des langages
%généralistes et des langages dédiés en intégrant des constructions spécialisées
%dans des outils généralistes. Cela nous permet d'ajouter une traçabilité dans
%une transformation écrite en {\java} notamment. Nous décrivons l'aspect cet
%aspect technique dans le chapitre~\ref{ch:outils:trace}, avec l'explication de
%l'implémentation des outils.

%\todo{en vrac}
%
%La traçabilité \emph{interne} est la traçabilité qui permet de réaliser la
%phase de résolution. Elle consiste à marquer les éléments cibles d'une
%\emph{définition} comme étant des éléments pouvant correspondre à des éléments
%\emph{resolve}. Cette traçabilité a un intérêt technique.
%
%\todo{en vrac}
%
%Dans un environnement parfaitement maîtrisé, il est relativement simple
%d'assurer la traçabilité d'une transformation : en effet, l'utilisateur a moins
%de possibilités de s'écarter de l'environnement. Dans notre approche hybride,
%nous sommes constamment à la frontière de deux mondes que l'utilisateur peut
%franchir selon ses convenances. Si l'utilisation du langage {\tom} et la
%manipulation de termes ne lui conviennent pas, l'utilisateur peut revenir à un
%mode de programmation en pur {\java}, auquel cas nous perdons le contrôle de
%cette partie du code. Ajouter la traçabilité dans cet environnement devient
%plus complexe. Nous avons pris le parti de fournir une traçabilité qui puisse
%être utilisée simplement au sein de {\java}.

%\todo{en vrac}
%
%Nous avons vu dans le chapitre~\ref{ch:verification:trace} que la traçabilité
%est généralement classée en deux catégories : \emph{implicite} et
%\emph{explicite}. Dans notre approche, nous avons fait le choix d'une
%traçabilité \emph{explicite}, ce qui implique l'introduction d'une syntaxe
%concrète dans le langage pour capturer explicitement les liens de trace.

%\todo{en vrac}
%
%Dans notre approche, afin d'obtenir des traces exploitables, nous avons choisi
%d'aborder le problème de la traçabilité sur une action explicite de
%l'utilisateur. Dans l'extension du langage {\tom}, nous proposons une
%construction dédiée à la traçabilité.

%\todo{en vrac}
%
%Nous distinguons deux types de traçabilité : la traçabilité dite \emph{interne}
%ou \emph{technique}~\cite{Jouault2005} ainsi que la traçabilité de
%spécification.

%\todo{en vrac}
%
%La traçabilité \emph{interne} a un usage purement technique au sein de la
%transformation, c'est-à-dire qu'elle est utilisée pour opérer la transformation
%mais n'est pas nécessairement utile dans le cadre d'un processus de
%qualification. Cette traçabilité est nécessaire à la réalisation de la phase de
%résolution. En effet, si l'on crée des éléments \emph{resolve}, il faut aussi
%marquer les éléments susceptibles de correspondre à un \emph{resolve} donné.
%Les éléments tracés de la sorte sont donc les pendants des \emph{resolve}.


%\section{Travail connexe (\todo{section à part entière ou au début plutôt ?})}

%\section{\todo{Synthèse}}
%\ttodo{}


% vim:spell spelllang=fr
