\begin{figure}[h!]

  % Define the layers to draw the diagram
  \pgfdeclarelayer{tomgombackground}
  \pgfdeclarelayer{tombackground}
  \pgfsetlayers{tomgombackground,tombackground,main}

  % Define a new shape: page with a folded corner 
  \makeatletter
  \pgfdeclareshape{flippedpage}{
    \inheritsavedanchors[from=rectangle] % this is nearly a rectangle
    \inheritanchorborder[from=rectangle]
    \inheritanchor[from=rectangle]{center}
    \inheritanchor[from=rectangle]{north}
    \inheritanchor[from=rectangle]{south}
    \inheritanchor[from=rectangle]{west}
    \inheritanchor[from=rectangle]{east}
    % ... and possibly more
    \backgroundpath{% this is new
    % store lower left in xa/ya and upper right in xb/yb
    \southwest \pgf@xa=\pgf@x \pgf@ya=\pgf@y
    \northeast \pgf@xb=\pgf@x \pgf@yb=\pgf@y
    % compute corner of ``flipped page''
    \pgf@xc=\pgf@xb \advance\pgf@xc by-5pt % this should be a parameter
    \pgf@yc=\pgf@yb \advance\pgf@yc by-5pt
    % diagonal path of the corner 
    \pgfpathmoveto{\pgfpoint{\pgf@xa}{\pgf@ya}}
    \pgfpathlineto{\pgfpoint{\pgf@xa}{\pgf@yb}}
    \pgfpathlineto{\pgfpoint{\pgf@xc}{\pgf@yb}}
    \pgfpathlineto{\pgfpoint{\pgf@xb}{\pgf@yc}}
    \pgfpathlineto{\pgfpoint{\pgf@xb}{\pgf@ya}}
    \pgfpathclose
    % add little corner
    \pgfpathmoveto{\pgfpoint{\pgf@xc}{\pgf@yb}}
    \pgfpathlineto{\pgfpoint{\pgf@xc}{\pgf@yc}}
    \pgfpathlineto{\pgfpoint{\pgf@xb}{\pgf@yc}}
    \pgfpathlineto{\pgfpoint{\pgf@xc}{\pgf@yc}}
    }
  }

  % Define block styles
  \tikzstyle{compiler}=[draw, fill=blue!20, text width=6em, text centered, minimum height=2.5em, rounded corners]
  %\tikzstyle{compiler}=[diamond, aspect=2, draw, fill=blue!20, text centered, rounded corners=1]
  \tikzstyle{compiler2}=[diamond, aspect=2, draw, text centered, rounded corners=1]
  %\tikzstyle{code}=[draw,shape=flippedpage,text width=2.7em, text centered, minimum height=3.7em,fill=white]
  \tikzstyle{code}=[draw,shape=flippedpage,text width=3em, text centered,
  minimum height=4.2em,fill=white]
  \tikzstyle{texttitle}=[fill=white, rounded corners, draw=black!50, dashed]
  \tikzstyle{background}=[fill=yellow!20, rounded corners, draw=black!50, dashed]

  \begin{center}
  \begin{tikzpicture}
    % Draw diagram elements
    
    \node (mapping) [code, dashed] {\figcode{Ancrages algébriques}};

    \path (mapping.east)+(3,-2.5) node (datastruct)[code] 
    {\figcode{Java EMF (MM)}};
    \path (mapping.west)+(-3,2.5) node (tomcode)[code, dashed] 
      {\figcode{Tom \\ + \\ Java}};
    \path (tomcode.east)+(3,0) node (tomcomp)[compiler2, dashed] 
    {\figcode{Compilateur {\tom}}};
    \path (mapping.east)+(3,2.5) node (javacode)[code] 
      {\figcode{Code Java}};
    \path (mapping.east)+(6,2.5) node (javacomp)[compiler2] 
    {\figcode{Compilateur Java}};
    \path (mapping.east)+(9,2.5) node (bytecode)[code]
      {\figcode{110010}\\\figcode{101110}\\\figcode{010100}};
    \path (mapping.east)+(7,0) node (trace)[code, dashed] 
      {\figcode{Trace ?}};

    % Draw gom side
    \path (datastruct.west)+(-3,0) node (emfcomp)[compiler2, dashed] 
    {\figcode{Tom-EMF}};

    % Draw arrows between elements
    \path [draw, ->] (emfcomp.north) -- node [below] {} (mapping);
    \path [draw, ->] (mapping.north) -- node [below] {} (tomcomp);
    \path [draw, ->] (datastruct.west) -- node [right] {} (emfcomp);
    \path [draw, ->] (datastruct.east) -- node [below] {} (javacomp);
    \path [draw, ->] (tomcode.east) -- node [left] {} (tomcomp);
    \path [draw, ->] (tomcomp.east) -- node [left] {} (javacode);
    \path [draw, ->] (javacode.east) -- node [left] {} (javacomp);
    \path [draw, ->] (javacomp.east) -- node [left] {} (bytecode);

  \end{tikzpicture}
  \end{center}
  \caption{Processus de compilation d'une transformation de modèles {\tom}+{\emf}}
\label{fig:tomEMFCompiler}
\end{figure}

